\section{Purpose of the meeting}

The main goal of this meeting was to fix a deadline to the next submission,
	Rafik and me proposed the iWCMC conference that will be organized in Lebanon during the last week of Jun of the next year.
The deadline is Jan $10^{th}$, 2020 (I realize now that time goes fast !).
If I could get all the required parameters and metrics to train better my algorithms and compare it with other approaches this could be worthy to be submitted to ICC.

\section{Main discussed points}

The main idea that we discussed during this meeting are elucidated below:
	
\subsection{Genetic algorithm for LoRa}

First,
	as I submitted a poster with 2 pages and the idea seems to be interesting for reviewers,
	the most important thing until the next submission is to apply a reinforcement learning and compare in with the genetic algorithm.
The main Challenge to achieve this is to get all the required metrics to be able to compare this two approaches with others.
As Nawel said,
	We can improve our model by asking users if they are satisfied by the QoS and get their answers and use them as an input to increase the QoE.

\subsection{Slice orchestration}

The second point that we discussed is to isolate traffic by the QoS required by each application.
In fact ``One size fits all'' problem is no longer bearable especially in industrial applications.
Industrial companies want to get an end-to-end control of their traffic and not only the transmission configuration.
So the need to classify traffic to different slices of traffic is approved in the literature as the unique solution \cite{chang_ran_2018}.

\subsection{Clustering algorithms}

The third point that we discussed was to use clustering algorithms to optimize the energy consumption of both the end devices and the gateways.
In fact,
	If all end-devices try to send their data to the network server based only on the RSSI of the gateway this could lead to increase traffic load,
	delay and packets loss.
Clustering algorithms take in consideration other parameters like the kind of traffic loaded, the location,
	the reputation of end devices and the required bandwidth.
Once the end-devices are clustered,
	those who belong to the same cluster could send their data through the same gateway.
One of the clustering algorithm that I found and could be useful in our case is \red{BIRCH (Balanced Iterative Reducing and Clustering using Hierarchies) Algorithm used in \cite{dawaliby_adaptive_2019}}.
It allows avoiding clusters with contains numerous end-devices that exceed the initial threshold,
	it is useful only if we have a lot of devices in the construction site.
\red{There is also LEACH (Low Energy Adaptive Clustering Hierarchy) algorithm used in \cite{juma_cooperative_nodate} to cluster devices which have the same energy requirement profile.
This algorithm has been tested in cellular network and BIRCH in LoRa network, we can test the accuracy of LEACH algorithm in LoRa and compare it with \cite{dawaliby_adaptive_2019}}.
\subsection{Tracking systems}

The fourth Point that we discussed together is the ability to track persons,
	vehicles and any other objects without the need of GPS.
GPS localization in an urban area could not give the required accuracy for tracking,
	it could be even worse in a construction site due to the noise generated by building machinery.
To overcome this limitation DGPS techniques could be used to get the right location of the base stations,
	devices that are beside a base station could get their location based on the location of the closest base station.
This solution requires additional hardware and could be expensive if more than one base station is required in the construction site.

\subsection{Objenious}

Objenious is a tool that collect all the data sent by end devices in a single platform.
It maps the end-devices to their location and gives information about the data collected by each device.
I will read more about this platform to see how it can be beneficial to us.

\subsection{Gateways position}

In a city and particularly in a construction site,
	finding the right position of gateways to cover all the construction site is a big challenge.
We talked about this problem and Laurent send us a thesis report that try to solve this problem to optimize the energy consumption.
\section{Conclusion}

We discussed a lot of points,
	all of them are interesting and the smart decision right now is to focus on the three first points to finish our first contribution.
The second contribution is to focus on the location of gateways and not only the traffic loaded to gateways.

The next meeting will be on December $3^{rd}$,
	we will discuss the application of reinforcement learning to select LoRa transmission parameters \cite{chincoli_self-learning_2018} \cite{bonnefoi_multiarmed_2017}.


%!TEX output_directory = bin

%\documentclass[a4paper]{article}

\input{\jobname}

\usepackage[papersize={8.5in,11in}, left=0.9in, right=0.9in, top=1in, bottom=1in]{geometry}

%\title{Projet PFE: Air pollution monitoring system using LoRa communication}
\title{\textbf{Projet PPE:} Evaluation Of Air Pollution Based On  Road Traffic Congestion in Urban Areas}
\date{Septembre 2019}

\begin{document}
%\setcounter{page}{0}
\maketitle
%\thispagestyle{fancy}


\section{Projet de thèse}

Aujourd'hui,
	les villes intelligentes disposent des réseaux sans fil hétérogènes dans le but de simplifier,
	d'améliorer et de sécuriser notre vie quotidienne.
Par exemple,
	les réseaux sans fil ad hoc pour véhicules (IEEE 802.11p) et les réseaux LTE (LTE-V) sont proposés pour améliorer la sécurité routière.
L'Internet industriel des objets (IIoT) est également une alternative pour les applications de sécurité et de divertissement.
Les réseaux de capteurs sans fil,
	tels que IEEE 802.15.4,
	LoRa et NB-IoT,
	sont des réseaux qui collectent et communiquent des données pour construire une vue globale de l'environnement de la ville.
Toutes ces données collectées peuvent être utilisées pour prédire les comportements humains et améliorer la sécurité de leurs déplacements grâce à des algorithmes d'apprentissage.
Ainsi,
	la fiabilité et la qualité de service de la transmission des données restent un des problèmes de recherche avec des contraintes différentes à chaque couche de communication.

La dynamique et l'hétérogénéité des appareils et de leur utilisation rendent la conception de réseaux robustes et évolutifs très complexe.
Cependant,
	les réseaux restreints (IIoT ou véhicules) sont vulnérables à plusieurs types d'attaques.
Par exemple,
	l'attaquant ou le brouilleur peut générer un bruit,
	interférant avec les fréquences radio utilisées par les dispositifs IoT.
Il peut en résulter un plus grand nombre de communications provenant d'appareils IoT pouvant épuiser leurs batteries.
Par conséquent,
	le déni de service pourrait facilement être causé.
Même si la sécurité est assurée,
	le réseau doit fournir une bonne qualité de service à d'autres applications (par exemple,
	les retards de transmission).
Les protocoles de qualité de service (QoS) peuvent permettre aux réseaux d'identifier et de hiérarchiser la transmission des données en fonction de la criticité des données.
Dans la littérature,
	plusieurs protocoles ont été proposés concernant la sécurité et la QoS \cite{simiscuka_relay_2018}.
Le protocole MQTT (Message Queuing Telemetry Transport) a été défini avec trois niveaux de QoS.
Il supporte la fiabilité des messages en définissant leurs priorités mais sans diminuer les délais sur l'ensemble des appareils du réseau.
La stratégie de défense contre les attaques de brouillage pourrait être basée sur l'évasion spectrale,
	le contrôle de la puissance de sortie ou le codage des communications en fonction des ressources des dispositifs \cite{xu_jamming_2006}\cite{stellios_survey_2018}.

\section{Objectifs et contributions attendues}

Un démonstrateur d'un système de contrôle des feux de circulation urbaine basé sur (IoT-UTLC) a été prototypé à ECE Paris.

\Itemize{
\item Le premier objectif de la thèse serait d'intégrer des aspects innovants à notre IoT-UTLC.
Le candidat étudiera de nouvelles architectures,
	de nouveaux protocoles et proposera plusieurs cas d'utilisation tenant compte du délai et de la sécurité de la transmission des données.

\item Le deuxième objectif sera l'étude des vulnérabilités de sécurité et des attaques possibles dans les réseaux IoT pour les villes intelligentes.
Cette étude portera sur deux niveaux d'attaques :
	les attaques physiques et les attaques de la couche application.

\item Le troisième objectif consistera à évaluer et à étudier l'immunité des réseaux du monde réel.
Le candidat proposera des solutions pour les vulnérabilités identifiées et démontrera leur efficacité.
}

\section{Tâches}
\Itemize{
	\item État de l'art du sujet proposé (Sécurité et QoS)
	\item Conception des solutions aux problèmes décrits
	\item Mise en œuvre, tests et évaluations 
}


\end{document}

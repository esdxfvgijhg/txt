\section{Approaches}

Quantifying and measuring privacy is very challenging,
	mainly because the definition of privacy is very subjective,
	each individual might have a different opinion about this concept.
In our work, we present privacy challenges through three points of view: behavioral, social and technical.

\subsection{Behavioral}

%Identifying Spam Without Peeking at the Contents
We consider two types of behavior models in this work,
	one based on typical behavior from a user perspective and one on typical behavior of a users' messages.
Both of these models can be quantified into a probability score.

%user perspective
\subsubsection{User behavior}

The user's behavior is modeled with respect to their typical service messaging usage,
	the frequency and type of messages received and sent,
	and the typical recipients with whom they exchange messages.
Known as a behavior profile,
	this type of model is computed over some training period to learn how the user behaves within the messaging account.

The measurements of the user's message behavior include the frequency of inbound/outbound message traffic,
	the specific times messages arrive and are sent,
	the "social cliques" of a user,
	and the user's response rates when replying to specific senders.

%1
Vidyalakshmi et al. \cite{vidyalakshmi_privacy_2015} proposed a privacy scoring using bezier curve.
They present a framework for calculating a privacy score metric considering \textbf{users’ personal attitude towards privacy and communication information}.
They focus on the rating of the user’s OSN friends based on their attitudes towards privacy,
	helping him to make an informed decision of sharing information with them.
Bezier curve in its cubic form is used as it has to account for both privacy orientation and communication orientation of the user.

\cite{alemany_estimation_2018}

\cite{zhang_privacypreserving_2017}

%2
%\cite{li_algorithm_2016} identify the seed node set to spread the information to have a better trade-off of utility and privacy cost.

%3
\cite{liu_framework_2010} propose a model to compute a privacy score of a user.
The privacy score increases based on how sensitive and visible a profile item is and can be used to adjust the privacy settings of friends.
Their solution also focused on the privacy settings of users with respect to their profile items.
They use Item Response Theory (IRT) to evaluate \textbf{sensitivity and visibility of attributes} when evaluating privacy scores.
The authors definition of privacy score satisfies the following intuitive properties: the more sensitive information a user discloses,
	the higher his or her privacy risk.
However,
	their approach do not support personalized privacy view over profile content for each individual in the social network.

%%Game theory
%Sallhammar et al. \cite{sallhammar_stochastic_2006} suggest the use of game theory as a method for computing the probabilities of \textbf{expected attacker behavior} in a quantitative stochastic model of security.
%By viewing system states as elements in a stochastic game,
%	they compute the probabilities of expected attacker behavior and model attacks as transitions between the system states.
%Having solved the game,
%	the expected attacker behavior is reflected in the transitions between the states in the system model,
%	by weighting the transition rates according to probability distributions.
%The proposed game model is based on a reward and cost concept and a detailed evaluation of how the reward and cost parameter influence the expected attacker behavior is included.
%In the final step,
%	continuous-time Markov chain (CTMC) is used to compute operational metrics of the system.

%There are lots of applications that act as an expert system.
%In such applications,
%	the existing user’s preference and actions are stored,
%	analyzed and used for giving relevant suggestions for naive users.
%The skeptical part of using these applications are about how they handle the user’s data.
%Many TPAs does not provide the required level of service that matches the user’s expectation,
%	in which case the user tends to uninstall the application and choose a new application.

%The user ends up sharing their information with many applications leaving silos of their personal information here and there,
%	creating an opportunity for the advertising agents and data aggregators to correlate such information,
%	and create a profile of the users.
%Many users are not aware of the extent of information they share with the TPA \\cite{shanmughapriya\_alert\_2016} conducted a survey to record the users desirable level of data to be shared,
%	and also captured the actual data shared by the user with TPA.














%message perspective
\subsubsection{Message behavior}

The second type of behavior is specific to how spam behaves and how it appears in the message folder among normal messages.
In general,
	spam messages can be easily detected because they appear anomalous with respect to the normal set of messages received and opened by the user.
However, ...
TODO

%4

%5

%6

%7

%8

%9

%10



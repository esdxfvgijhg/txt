\subsection{Social (OSN)}

\subsubsection{Concept of privacy}

% A Study of Online Social Network Privacy Via the TAPE Framework
Privacy in online social network (OSN) have attracted many attentions.
OSN service providers allow users to manage who can access which information and communication (e.g. Facebook and Google+).
Researcher studied privacy protection from two directions.
Along the first direction,
	fundamental changes to the current design of OSN were suggested to enhance users' privacy.
Within this direction, "Privacy by Design" (PbD) is an important approach.
For example,
	in \cite{baden_persona:_2009},
	Baden et al. proposed a new type of OSNs by using attribute-based encryption to hide user data,
	in which symmetric keys are used to encrypt messages and only the designated friend groups can decrypt the messages.
In \cite{erkin_generating_2011},
	Erkin et al. proposed to use homomorphic encryption and multi-party computation techniques to hide privacy-sensitive data from the service provider in a recommender system, 

The second direction is developing privacy protection tools based on existing OSNs.
In this section we focus on the second direction to deal with current OSNs.

%1
\cite{akcora_profiling_2014} focuses on on the risk of new interactions from a privacy point of view.

%2
\cite{b.s._privacy_2015} proposed a privacy control framework for information dispersal on social network,
	they use the quadratic form of bezier curve to arrive at privacy scores for friends,
	they use the communication information for pre-sorting of friends which is lacking in \cite{vidyalakshmi_privacy_2015}.

%3
Privacy Index (PIDX) proposed in \cite{nepali_sonet:_2013} is a measure of a user’s privacy exposure in a social network.
PIDX is a numerical value between 0 and 100 with high value indicating high privacy risk in social networks.
Each attributes privacy impact factor is the ratio of its privacy impact to full privacy disclosure,
	is the summation of privacy impact factors of each attribute visible.

%4
\cite{bilogrevic_multi-dimensional_2014} study the privacy of Social Relationships in Pervasive Networks.

%5
\cite{akcora_risks_2012} develop a graph-based approach and a risk model to learn risk labels of strangers with the intuition that risky strangers are more likely to violate privacy constraints.

%6
Relationship between user’s trustworthiness and privacy risk is presented in \cite{pandey_computing_2015}.

%7

%8
In \cite{gundecha_exploiting_2011},
	Sun et al proposed a probability trust model that uses Beta function to address concatenation propagation and multi-path propagation of trust.

%9
Zeng and Xing \cite{yongbo_zeng_study_2015} studied how individual users can expand their social networks by making trustful friends who will not leak their private IFs to unknown parties.
\textbf{Trust and reputation concepts} are used in order to preserve user’s privacy while increasing their social capital in OSNs.
Here it has a balance point of social capital and privacy thresholds that maximizes correct IF diffusion while minimizing illegal private IF leak out,
	given user’s risk appetite for preserving privacy.

%9 TAPE
Zeng et al. \cite{yongbo_zeng_study_2015} approaches the privacy quantification problem from a different angle.
First,
	they consider how likely a friend reveals others’ personal information,
	by computing the privacy trust score,
	which is a widely studied research problem \cite{gundecha_exploiting_2011}.
Next,
	the proposed work is related to information diffusion in OSNs.
For example,
	researchers attempt to build mathematical model to solve problems of information diffusion in OSNs,
	such as \cite{fang_privacy_2010}.
Finally,
TAPE framework differs from other work,
	in considering information diffusion in the context of privacy protection,
	which requires different sets of features and considerations.

%10

\subsubsection{Concept of reputation}

%SNARE
SNARE \cite{hao_detecting_2009} infers the reputation of an email sender solely based on network-level features,
	without looking at the contents of a message.
Using an automated reputation engine,
	SNARE classifies email senders as spammers or legitimate with about a 70\% detection rate for less than a 0.3\% false positive rate.
However,
	lacking authentication and non-repudiation in standard trust and reputation solution make these solutions be subject to identity spoofing,
	false accusation and collusion attacks.
Further,
	these solutions consume extra valuable resources of email servers on email reception and filtering.

\subsubsection{Concept of trust}

%Trust-involved access control in collaborative open social networks
Trust in a person is a commitment to an action based on a belief that the future actions of that person will lead to a good outcome.
There are three main properties of trust that are relevant to the development of algorithms for computing it \cite{wang_trust-involved_2010},
	namely,
	transitivity,
	asymmetry,
	and personalization.
%transitivity
The primary property of trust that is used in our work is transitivity.
	if Alice highly trusts Bob,
	and Bob highly trusts Chuck,
	it does not always and exactly follow that Alice will highly trust Chuck.
%asymmetry
It is also important to note the asymmetry of trust,
	for two people involved in a relationship,
	trust is not necessarily identical in both directions.
%personalization
The third property of trust that is important in social networks is the personalization of trust,
	trust is inherently a personal opinion,
	two people often have very different opinions about the trustworthiness of the same person.
%%%%%%%%%%%%%%%%%%%%%%%%%%%%%%%%%%%%%%%%%%%%%%%%%%%%%%%%%%%%%%%%%%%%

%Ostra
Ostra \cite{mislove_ostra:_2008} utilizes trust relationship to thwart unwanted communication,
	where the number of a user’s trust relationships is used to limit the amount of unwanted communications he can produce.
Ostra relies on existing trust networks to connect senders and receivers via chains of pair-wise trust relationship and use a pair-wise,
	link-based credit scheme to impose a cost on originator of unwanted communication.
However,
	its scalability of this system stays uncertain as it employs a per-link credit scheme.

%
In \cite{zeng_trust-aware_2014},
	Sun et al proposed a probability trust model that uses Beta function to address concatenation propagation and multi-path propagation of trust.

%LENS: Leveraging Social Networking and Trust to Prevent Spam Transmission

%SOAP
SOAP \cite{li_soap:_2011} presents a social network based personalized spam filter that integrates social closeness,
	user (dis)interest and adaptive trust management into a Bayesian filter.
However,
	several issues with SOAP,
	including the intrinsic cost of initialization and continuous adaptation of social closeness (between sender and recipient) 
		and social interests (of an individual) in the Bayesian filter,
	limit its usage.

%SocialFilter
SocialFilter \cite{yang_socialfilter:_2009} proposes a collaborative spam mitigation system that uses social trust embedded in OSN to asses the trustworthiness of Spam reporter.
The spammer reports from the SocialFilter nodes are stored at a centralized repository that computes the trust values of the reports and identifies spammers based on IP addresses.
However,
	the SocialFilter’s effectiveness is doubtful as spammers may use dynamic IPs.

%%Social Market: Combining Explicit and Implicit Social Networks

%SybilGuard
SybilGuard [23] and SybilLimit [22] propose protocols that exploit trust relationships between friends to protect peer-to-peer systems from sybil attacks.

%NABT
NABT [15] proposes the use of trust between friends to prevent free-riding behaviors using an indirect trust relationships,
	NABT’s credit-based approach can be viewed as a basic form of trust inference between friends of friends.
A more advanced approach to trust-inference is adopted by SUNNY [14],
	a centralized protocol that takes into account both trust and confidence to build a Bayesian network.

%TrustWalker
TrustWalker [10] combines trust and item-based collaborative filtering.

Trust metrics can be classified to two main categories: \textbf{global and local trust metrics}.
%Global trust metrics
Global trust metrics [7] predict a global reputation value for each node. %PageRank
%Local trust metrics
Local trust metrics [16],
	on the other hand,
	compute trust values that are dependent on the target user.
Local trust metrics take into account the very personal and subjective views of the users and predict different values of trust in other users for every single user.
%TaRS
TaRS \cite{massa_trust-aware_2007} builds a recommendation system capable of operating both global and local trust metrics.

%Social Market
Despite the mole of work on social trust,
	Social Market is,
	to the best of our knowledge,
	the first system to propose the use of trust relationships to build a decentralized interest-based marketplace.
Similarly,
	TAPS is the first attempt to combine explicit and implicit social networks into a single gossip protocol.

\paragraph{Trust networks and trust metrics}
The system can ask the users to rate other users: in this way,
	a user can express her level of trust in another user she has interacted with,
	i.e. issue a trust statement such as “I, Alice, trust Bob as 0.8 in [0,1]”.
The system can then aggregate all the trust statements in a single trust networks representing the relationships between users.

Trust metrics are algorithms whose goal is to predict,
	based on the trust network,
	the trustworthiness of “unknown” users,
	i.e. users in which a certain user didn’t express a trust statement.
Their aim is to reduce social complexity by suggesting how much an unknown user is trustworthy.

\subsubsection{Concept of collaboration}

%Privacy and Social Capital in Online Social Networks

The concept of trust is used to indicate the relationship between two entities,
Reputation concept is used to refer to a more general sense of trust towards a particular entity based on opinions of multiple entities \cite{cho_survey_2015, hussain_overview_2007}.

%Trust-involved access control in collaborative open social networks
%collaborative community
A possible solution is to estimate the trust level to be assigned to a user in a collaborative community on the basis of his/her reputation,
	given by his/her behavior with regards to all the other users in the community.
In our scenario,
	this can be done by making a user able to monitor the behavior of the other users wrt the release of private

%Detecting and Resolving Privacy Conflicts for collaborative Data Sharing in Online Social Networks
Hu et al. \cite{hu_detecting_2011} propose an approach to enable collaborative privacy management of shared data in OSNs.
In particular,
	they provide a systematic mechanism to identify and resolve privacy conflicts for collaborative data sharing.
their conflict resolution indicates a trade-off between privacy protection and data sharing by quantifying privacy risk and sharing loss

%COAT : Collaborative Outgoing Anti-Spam Technique
%community collaboration
The collaborative systems [4] do not rely upon semantic analysis but on the community to identify spam messages.
Once a message is tagged as spam by one SMTP server,
	the signature of that message is transmitted to all other SMTP servers.
This class requires the collaboration of multiple SMTP servers to implement the system.



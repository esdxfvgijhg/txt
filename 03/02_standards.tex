\section{IoT Wireless Networks (Norms \& Standards)}

% \Figure{!htb}{.5}{riot-gnrc-stack.svg}
%\Figure{!htb}{1}{lpwan.pdf}{LPWAN}


Several different wireless communication protocols,
	such as Wireless LAN (WLAN),
	BLE, 6LoWPAN,
	and ZigBee may be suitable for IoT applications.
They all operate in the 2.4GHz frequency band and this,
	together with the limited output power in this band,
	means that they all have a similar range.
The main differences are located in the MAC,
	PHY,
	and network layer.
WLAN is much too power hungry as seen in table 2.6 and is only listed as a reference for the comparisons.


\subsection{SigFox}

\subsection{IETF}
\subsubsection{6LoWPAN} 
is a relatively new protocol that is maintained by the Internet
Engineering Task Force (IETF) [7, 6].
The purpose of the protocol is to enable IPv6 traffic over a IEEE 802.15.4 network with as low overhead as possible;
	this is achieved by compressing the IPv6 and UDP header.
A full size IPv6 + UDP header is 40+8 bytes which is tild 38\% of a IEEE 802.15.4
frame,
	but with the header compression this overhead can be reduced to 7 bytes,
	thus reducing the overhead to tild 5\%,
	as seen in figures 2.3 and 2.4.


\subsection{3GPP}
\subsubsection{NB-IoT}
\subsubsection{EC-GSM}
\subsubsection{e-MTC}


\subsection{IEEE}
\subsubsection{IEEE 802.11}



\subsubsection{IEEE 802.15.4}
At present days,
	there are several technology standards.
Each one is designed for a specific need in the market.
For the Wireless Sensor Networks,
	the aim is to transmit little information,
	in a small range,
	with a small power consumption and low cost.
The IEEE 802.15.4 standard offers physical and media access control layers for low-cost,
	low-speed,
	low-power Wireless Personal Area Networks (WPANs)

\paragraph{Physical Layer}
The standard operates in 3 different frequency bands:
- 16 channels in the 2.4GHz ISM band
- 10 channels in the 915MHz ISM band
- 1 channel in the European 868MHz band


\paragraph{Definitions}
Coordinator:
	A device that provides synchronization services through the transmission of beacons.
PAN Coordinator:
	The central coordinator of the PAN.
This device identifies its own network as well as its configurations.
There is only one PAN Coordinator for each network.
Full Function Device (FFD):
	A device that implements the complete protocol set,
	PAN coordinator capable ,
	talks to any other device.
This type of device is suitable for any topology.
Reduced Function Device (RFD):
	A device with a reduced implementation of the protocol,
	cannot become a PAN Coordinator.
This device is limited to leafs in some topologies.

\paragraph{Topologies}
Star topology:
	All nodes communicate via the central PAN coordinator ,
	the leafs may be any combination of FFD and RFD devices.
The PAN coordinator usually uses main power.

Peer to peer topology:
	Nodes can communicate via the central PAN coordinator and via additional point-to-point links .
All devices are FFD to be able to communicate with each other.


Combined Topology:
	Star topology combined with peer-to-peer topology.
Leafs connect to a network via coordinators (FFDs) .
One of the coordinators serves as the PAN coordinator .



\subsubsection*{IEEE 802.15.4}
The IEEE 802.15.4 standard defines the PHY and MAC layers for wireless communication [6].
It is designed to use as little transmission time as possible but still have a decent payload,
	while consuming as little power as possible.
Each frame starts with a preamble and a start frame delimiter;
	it then continues with the MAC frame length and the MAC frame itself as seen in figure 2.2.
The overhead for each PHY packet is only 4+1+1 133 tild 4.5\%;
	when
using the maximum transmission speed of 250kbit/s,
	each frame can be sent 133byte in 250kbit/s = 4.265ms.
Furthermore,
	it can also operate in the 868MHz and 915MHz bands,
	maintaining the 250kbit/s transmission rate by using Offset quadrature phase-shift keying (O-QPSK).

Several network layer protocols are implemented on top of IEEE 802.15.4.
The two that will be examined are 6LoWPAN and ZigBEE.

\subsubsection{ZigBee} 
is a communication standard initially developed for home automation
networks; it has several different protocols designed for specific areas such
as lighting, remote control, or health care [27, 6]. Each of these protocols
uses their own addressing with different overhead; however, there is also the
possibility of direct IPv6 addressing. Then, the overhead is the same as for
uncompressed 6LoWPAN, as seen in figure 2.5.

A new standard called ZigBee 3.0 aims to bring all these standards together under one roof to simplify the integration into IoT.
The release date of this standard is set to Q4 2015.

\subsection{LoRaWAN}
%Low Power Wide Area Networks
\Figure{h}{1}{lora_stack.png}{uhuhuh}


LoRa (Long Range) is a proprietary spread spectrum modulation technique by Semtech.
It is a derivative of Chirp Spread Spectrum (CSS).
The LoRa physical layer may be used with any MAC layer;
	however,
	LoRaWAN is the currently proposed MAC which operates a network in a simple star topology.

As LoRa is capable to transmit over very long distances it was decided that LoRaWAN only needs to support a star topology.
Nodes transmit directly to a gateway which is powered and connected to a backbone infrastructure.
Gateways are powerful devices with powerful radios capable to receive and decode multiple concurrent transmissions (up to 50).
Three classes of node devices are defined:
	(1) Class A enddevices:
	The node transmits to the gateway when needed.
After transmission the node opens a receive window to obtain queued messages from the gateway.
(2) Class B enddevices with scheduled receive slots:
	The node behaves like a Class A node with additional receive windows at scheduled times.
Gateway beacons are used for time synchronisation of end-devices.
(3) Class C end-devices with maximal receive slots:
	these nodes are continuous listening which makes them unsuitable for battery powered operations.


\subsubsection{ALIANCE}


\paragraph{Class-A}

\subparagraph{Uplink}
LoRa Server supports Class-A devices.
In Class-A a device is always in sleep mode,
	unless it has something to transmit.
Only after an uplink transmission by the device,
	LoRa Server is able to schedule a downlink transmission.
Received frames are de-duplicated (in case it has been received by multiple gateways),
	after which the mac-layer is handled by LoRa Server and the encrypted application-playload is forwarded to the application server.

\subparagraph{Downlink}
LoRa Server persists a downlink device-queue for to which the application-server can enqueue downlink payloads.
Once a receive window occurs,
	LoRa Server will transmit the first downlink payload to the device.

\subparagraph{Confirmed data}
LoRa Server sends an acknowledgement to the application-server as soon one is received from the device.
When the next uplink transmission does not contain an acknowledgement,
	a nACK is sent to the application-server.

\textbf{Note:}
	After a device (re)activation the device-queue is flushed.

\Figure{!htb}{.8}{class_a.png}{Class A}



\paragraph{Class-B}
LoRa Server supports Class-B devices.
A Class-B device synchronizes its internal clock using Class-B beacons emitted by the gateway,
	this process is also called a “beacon lock”.
Once in the state of a beacon lock,
	the device negotioates its ping-interval.
LoRa Server is then able to schedule downlink transmissions on each occuring ping-interval.

\subparagraph{Downlink}
LoRa Server persists all downlink payloads in its device-queue.
When the device has acquired a beacon lock,
	it will schedule the payload for the next free ping-slot in the queue.
When adding payloads to the queue when a beacon lock has not yet been acquired,
	LoRa Server will update all device-queue to be scheduled on the next free ping-slot once the device has acquired the beacon lock.

\subparagraph{Confirmed data}
LoRa Server sends an acknowledgement to the application-server as soon one is received from the device.
Until the frame has timed out,
	LoRa Server will wait with the transmission of the next downlink Class-B payload.

\textbf{Note:} The timeout of a confirmed Class-B downlink can be configured through the device-profile.
This should be set to a value less than the interval between two ping-slots.

\subparagraph{Requirements}

\begin{itemize}
	\item[Device]
	The device must be able to operate in Class-B mode.
This feature has been tested against the develop branch of the Semtech LoRaMac-node source.
	\item[Gateway]
	The gateway must have a GNSS based time-source and must use at least the Semtech packet-forwarder version 4.0.1 or higher.
It also requires LoRa Gateway Bridge 2.2.0 or higher.

\end{itemize}

\Figure{!htb}{1}{class_b.png}{Class B}

\paragraph{Class-C}

\subparagraph{Downlink}
LoRa Server supports Class-C devices and uses the same Class-A downlink device-queue for Class-C downlink transmissions.
The application-server can enqueue one or multiple downlink payloads and LoRa Server will transmit these (semi) immediately to the device,
	making sure no overlap exists in case of multiple Class-C transmissions.

\subparagraph{Confirmed data}
LoRa Server sends an acknowledgement to the application-server as soon one is received from the device.
Until the frame has timed out,
	LoRa Server will wait with the transmission of the next downlink Class-C payload.

\textbf{Note:} The timeout of a confirmed Class-C downlink can be configured through the device-profile.

\Figure{!htb}{1}{class_c.png}{Class C}

% \subsection{Summary and discussion}


\subsubsection{SEMTECH}

\Figure{!htb}{1}{lorawan_parameters.png}{LoraWan Parameters}
\Figure{!htb}{1}{loraparam}{}


LoRa has four configurable parameters:
\Itemize{
	\item[BW] \textbf{Bandwidth:}
Bandwidth (BW) is the range of frequencies in the transmission band.
Higher BW gives a higher data rate (thus shorter time on air),
	but a lower sensitivity (due to integration of additional noise).
A lower BW gives a higher sensitivity,
	but a lower data rate.
Lower BW also requires more accurate crystals (less ppm).
Data is send out at a chip rate equal to the bandwidth.
So,
	a bandwidth of 125 kHz corresponds to a chip rate of 125 kcps.
The SX1272 has three programmable bandwidth settings: 500 kHz, 250 kHz and 125 kHz.
The Semtech SX1272 can be programmed in the range of 7.8 kHz to 500 kHz,
	though bandwidths lower than 62.5 kHz requires a temperature compensated crystal oscillator (TCXO).

\Figure{h}{.5}{bw.png}{}

	\item[CF] \textbf{Carrier Frequency:}
Carrier Frequency (CF) is the centre frequency used for the transmission band.
For the SX1272 it is in the range of 860 MHz to 1020 MHz,
	programmable in steps of 61 Hz.
The alternative radio chip Semtech SX1276 allows adjustment from 137 MHz to 1020 MHz.

	\item[CR] \textbf{Coding Rate:}
Coding Rate (CR) is the FEC rate used by the LoRa modem and offers protection against bursts of interference.
A higher CR offers more protection,
	but increases time on air.
Radios with different CR (and same CF/SF/BW),
	can still communicate with each other.
CR of the payload is stored in the header of the packet,
	which is always encoded at 4/8.

	\item[SF] \textbf{Spreading Factor:}
SF is the ratio between the symbol rate and chip rate.
A higher spreading factor increases the Signal to Noise Ratio (SNR),
	and thus sensitivity and range,
	but also increases the air time of the packet.
The number of chips per symbol is calculated as 2 sf .
For example,
	with an SF of 12 (SF12) 4096 chips/symbol are used.
Each increase in SF halves the transmission rate and,
	hence,
	doubles transmission duration and ultimately energy consumption.
Spreading factor can be selected from 6 to 12.
SF6,
	with the highest rate transmission,
	is a special case and requires special operations.
For example,
	implicit headers are required.
Radio communications with different SF are orthogonal to each other and network separation using different SF is possible.

	\item[Tx] \textbf{Transmition power:}

	\item[Payload] \textbf{Payload legth:}

}

% LoRa has four metrics performance:




\setlength\tabcolsep{5pt}
\setlength\hskip{0pt}

\begin{table}[h!]
\scriptsize
	\begin{tabular}{|P|P?P|P|P|P|P|P?P|P|P|P|P|P?P|P|P|P|P|P?}
	% \begin{tabular}{|P|P?P|P|P|P|P|P?P|P|P|P|P|P?P|P|P|P|P|P?}
	\textbf{SF} &                      & 07 & 08 & 09 & 10 & 11 & 12 & 07 & 08 & 09 & 10 & 11 & 12 & 07 & 08 & 09 & 10 & 11 & 12\\\cmidrule[1pt]{1-20}
	\           & \textbf{BW}          & \multicolumn{6}{c|}{125}     & \multicolumn{6}{c|}{250}     & \multicolumn{6}{c|}{500} \\\cmidrule[1pt]{1-20}
	07          & \multirow{6}{*}{125} & x  &    &    &    &    &    &    &    & x  &    &    &    &    &    &    &    & x  &   \\\cline{3-20}
	08          &                      &    & x  &    &    &    &    &    &    &    & x  &    &    &    &    &    &    &    & x \\\cline{3-20}
	09          &                      &    &    & x  &    &    &    &    &    &    &    & x  &    &    &    &    &    &    &   \\\cline{3-20}
	10          &                      &    &    &    & x  &    &    &    &    &    &    &    & x  &    &    &    &    &    &   \\\cline{3-20}
	11          &                      &    &    &    &    & x  &    &    &    &    &    &    &    &    &    &    &    &    &   \\\cline{3-20}
	12          &                      &    &    &    &    &    & x  &    &    &    &    &    &    &    &    &    &    &    &   \\\cmidrule[1pt]{1-20}

	07          & \multirow{6}{*}{250} &    &    &    &    &    &    & x  &    &    &    &    &    &    &    & x  &    &    &   \\\cline{3-20}
	08          &                      &    &    &    &    &    &    &    & x  &    &    &    &    &    &    &    & x  &    &   \\\cline{3-20}
	09          &                      & x  &    &    &    &    &    &    &    & x  &    &    &    &    &    &    &    & x  &   \\\cline{3-20}
	10          &                      &    & x  &    &    &    &    &    &    &    & x  &    &    &    &    &    &    &    & x \\\cline{3-20}
	11          &                      &    &    & x  &    &    &    &    &    &    &    & x  &    &    &    &    &    &    &   \\\cline{3-20}
	12          &                      &    &    &    & x  &    &    &    &    &    &    &    & x  &    &    &    &    &    &   \\\cmidrule[1pt]{1-20}

	07          & \multirow{6}{*}{500} &    &    &    &    &    &    &    &    &    &    &    &    & x  &    &    &    &    &   \\\cline{3-20}
	08          &                      &    &    &    &    &    &    &    &    &    &    &    &    &    & x  &    &    &    &   \\\cline{3-20}
	09          &                      &    &    &    &    &    &    & x  &    &    &    &    &    &    &    & x  &    &    &   \\\cline{3-20}
	10          &                      &    &    &    &    &    &    &    & x  &    &    &    &    &    &    &    & x  &    &   \\\cline{3-20}
	11          &                      & x  &    &    &    &    &    &    &    & x  &    &    &    &    &    &    &    & x  &   \\\cline{3-20}
	12          &                      &    & x  &    &    &    &    &    &    &    & x  &    &    &    &    &    &    &    & x \\\cmidrule[1pt]{1-20}
	\end{tabular}
\caption{\label{tab:uyuy} uyuyuy}
\end{table}


\begin{tabular} { l | l l l l l l } \\
	\           & $SF_{7}$ & $SF_{B}$ & $SF_{9}$ & $SF_{10}$ & $SF_{11}$ & $SF_{12}$ \\
	$SF_ {7}$   & { - 6 }         & { 16 }          & { 18 }          & { 19 }           & { 19 }           & { 20 } \\
	$SF_{8}$    & { 24 }          & { - 6 }         & { 20 }          & { 22 }           & { 22 }           & { 22 } \\
	$SF_ {9}$   & { 27 }          & { 27 }          & { - 6 }         & { 23 }           & { 25 }           & { 25 } \\
	$SF_{10}$   & { 30 }          & { 30 }          & { 30 }          & { - 6 }          & { 26 }           & { 28 } \\
	$SF_ {11}$  & { 33 }          & { 33 }          & { 33 }          & { 33 }           & { - 6 }          & { 29 } \\
	$S F_{12}$  & { 36 }          & { 36 }          & { 36 }          & { 36 }           & { 36 }           & { - 6 }
\end{tabular}
	

\begin{table}[h!]
\scriptsize
	\begin{tabulary}{\columnwidth}{L|L|L|L}
	\bf{Module}            & \bf{SX1261/62/68} & \bf{SX1272/73} & \bf{SX1276/77/78/79} \\\hline
	\bf{Modem}             & LoRa \& FSK       & LoRa           & LoRa                 \\\hline
	\bf{Link budget}       & 170dB             & 157dB          & 168dB                \\\hline      
	\bf{Power amplifier}   & /61: +15dBm       & +14 dBm        & +14dBm               \\      
	\                      & 62/68:+22dBm      &                &                      \\\hline                     
	\bf{Rx current}        & 4.6 mA            & 10 mA          & 10 mA                \\\hline      
	\bf{Bit rate}          & 62.5kbps-LoRa     & 300 kbps       & 300 kbps             \\   
	\                      & 300kbps-FSK       &                &                      \\\hline                     
	\bf{Sensitivity}       & -148 dBm          & -137 dBm       & -148 dBm             \\\hline      
	\bf{Blocking immunity} & 88 dB             & 89 dB          & Excellent            \\\hline      
	\bf{Frequency}         & 150-960 MHz       & 860-1000 MHz   & 137-1020 MHz         \\\hline      
	\bf{RSSI}              &                   & 127 dB         & 127 dB               \\\hline      
	\end{tabulary}
\caption{\label{tab:terdjfy} \cite{gaddam_comparative_2018}}
\end{table}



\subsection{Divers}
\subsubsection{IPLC}
\subsubsection{BACnet}
\subsubsection{Z-WAze}
\subsubsection{Bluetooth LE}
BLE is developed to be backwards compatible with Bluetooth,
	but with lower data rate and power consumption [28].
Featuring a data rate of 1Mbit/s with a peak current consumption less than 15mA,
	it is a very efficient protocol for small amounts of data.
Each frame can be transmitted 47bytes in 1Mbit/s = 376Mus;
	thanks to the short transmission time,
	the transceivers consumes less power as the transceiver can be in receive mode or completely off most of the time.
BLE uses its own addressing methods and as the MAC frame size (figure 2.6) is only 39bytes,
	thus IPv6 addressing is not possible.

Starting from Bluetooth version 4.2,
	there is support for IPv6 addressing with the Internet Protocol Support Profile;
	the new version allows the BLE frame to be variable between 2 257 bytes.
The network set-up is controlled by the standard Bluetooth methods,
	whereas IPv6 addressing is handled by 6LoWPAN as specified in IPv6 over Bluetooth Low Energy [29].



\subsection{Summary and discussion}


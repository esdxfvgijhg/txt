\newpage

\section{SDN platforms}

\begin{table}[h]
\scriptsize
	\begin{tabulary}{\columnwidth}{L|L|L}
	\textbf{Plan de controle} & \textbf{Plan de gestion}       & \textbf{Plan de doonées}   \\\hline
	Controle d'admission      & Controle et supervision de QoS & Controle du trafic         \\
	Réservation de ressources & Gestion de contrats            & Façonnage du trafic        \\
	Routage                   & QoS mapping                    & Controle de congestion     \\
	Signalisation             & Politique de QoS               & Classification de paquets  \\
	\                         &                                & Marquage de paquets        \\
	\                         &                                & Ordonnancements des paquets\\
	\                         &                                & Gestion de files d'attente \\
	\end{tabulary}
	\caption{\label{tab:qos} An example table.}
\end{table}

\begin{itemize}
	\item[\cite{qin_software_2014}] Many studies have identified \green{SDN} as a potential solution to the WSN challenges,
	as well as a model for \red{heterogeneous} integration.
	\item[\cite{qin_software_2014}] This \red{shortfall} can be resolved by using the \green{SDN approach.}
	\item[\cite{kobo_survey_2017}] \green{SDN} also enhances better control of \red{heterogeneous} network infrastructures.
	\item[\cite{kobo_survey_2017}] Anadiotis et al. define a \green{SDN operating system for IoT} that integrates SDN based WSN \textbf{(SDN-WISE)}.
		This experiment shows how \red{heterogeneity} between different kinds of SDN networks can be achieved.
	\item[\cite{kobo_survey_2017}] In cellular networks,
		OpenRoads presents an approach of introducing \green{SDN} based \red{heterogeneity} in wireless networks for operators.
	\item[\cite{ndiaye_software_2017}] There has been a plethora of (industrial) studies \green{synergising SDN in IoT}.
			The major characteristics of IoT are low latency,wireless access, mobility and \red{heterogeneity}.
	\item[\cite{ndiaye_software_2017}] Thus a bottom-up approach application of \green{SDN} to the realisation of \red{heterogeneous IoT} is suggested.
	\item[\cite{ndiaye_software_2017}] Perhaps a more complete IoT architecture is proposed,
			where the authors apply \green{SDN} principles in IoT \red{heterogeneous} networks.
	\item[\cite{bera_softwaredefined_2017}] it provides the \green{SDWSN} with a proper model of network management,
			especially considering the potential of \red{heterogeneity} in SDWSN.
	\item[\cite{bera_softwaredefined_2017}] We conjecture that the \green{SDN paradigm} is a good candidate to solve the \red{heterogeneity} in IoT.
\end{itemize}

\begin{table}[h!]
\scriptsize
	\begin{tabulary}{\columnwidth}{L|L|C|C|C|C|C}
	\textbf{Management architecture}                 & \textbf{Management feature}            & \textbf{Controller configuration} & \textbf{Traffic Control} & \textbf{Configuration and monitoring} & \textbf{Scapability and localization} & \textbf{Communication management}\\\hline
	\textbf{\cite{luo_sensor_2012} Sensor Open Flow} & SDN support protocol                   & Distributed                       & in/out-band              & \ok                                   & \ok                                   & \ok                              \\\hline
	\textbf{\cite{costanzo_software_2012} SDWN}      & Duty sycling, aggregation, routing     & Centralized                       & in-band                  & \ok                                   &                                       & \\\hline
	\textbf{\cite{galluccio_sdnwise_2015} SDN-WISE}  & Programming simplicity and aggregation & Distributed                       & in-band                  &                                       & \ok                                   & \\\hline
	\textbf{\cite{degante_smart_2014} Smart}         & Efficiency in resource allocation      & Distributed                       & in-band                  &                                       & \ok                                   & \\\hline
	\textbf{SDCSN}                                   & Network reliability and QoS            & Distributed                       & in-band                  &                                       & \ok                                   & \\\hline
	\textbf{TinySDN}                                 & In-band-traffic control                & Distributed                       & in-band                  &                                       & \ok                                   & \\\hline
	\textbf{Virtual Overlay}                         & Network flexibility                    & Distributed                       & in-band                  &                                       & \ok                                   & \\\hline
	\textbf{Context based}                           & Network scalability and performance    & Distributed                       & in-band                  &                                       & \ok                                   & \\\hline
	\textbf{CRLB}                                    & Node localization                      & Centralized                       & in-band                  &                                       &                                       & \\\hline
	\textbf{Multi-hope}                              & Traffic and energy control             & Centralized                       & in-band                  &                                       &                                       & \ok                              \\\hline
	\textbf{Tiny-SDN}                                & Network task measurement               & -                                 & in-band                  &                                       &                                       & \\
	\end{tabulary}
	\caption{\label{tab:Tableuy} SDN-based network and topology management architectures. \cite{ndiaye_software_2017}}
\end{table}
\twocolumn

\section{Blockchain}
\subsection{Application}
Blockchain Layers
\begin{itemize}
	\item Transaction \& contract layer
	\item Validation layer (forward validation request)
	\item Block Generation Layer (PoW,PoC, PoA PoS, PBFT)
	\item Distribution Layer\\
\end{itemize}

Consensus algorithms
\begin{itemize}
	\item Proof of Work (PoW)
	\item Proof of Capacity (PoC)
	\item Proof of Authority (PoA)
	\item Proof of Stake (PoS)
	\item Proof of Bizantine Fault Tolerant (PBFT)
\end{itemize}

\subsection{Summary and discussion}

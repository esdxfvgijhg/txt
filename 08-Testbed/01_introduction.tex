\section{Introduction \cite{bregell_hardware_2015}}

\subsection{Problem Statement}


\subsection{Background}

Internet of Things (IoT) is a concept aiming at connecting all things to the Internet [1].
The different kinds of devices range from simple sensor devices to complex machines such as industry robots.
Home automation has been available for a few years in the forms of timers and remotely controlled devices,
	such as lights,
	garage door,
	and climate control equipment.
Also in the industry and workplace,
	there are current systems that have some of the functionality of IoT,
	e.g,
	sensors in robots and machines which keep track of the system status so that maintenance can be scheduled at the right time.
However,
	these systems or sensors rarely communicate with each other or make decisions based on other sensor values;
	instead they depend on input from a user.
In the same way cellphones connected people and made them constantly connected to the Internet,
	IoT will connect devices and make them constantly connected to the Internet [2].
In theory,
	this could lead to a future with autonomous technology all around us.
The benefits could be huge as it would save time and energy for both the individual at home and for the industry [3].
IoT could be used in industry to automate power-heavy tasks to run when the electricity price is low.
This principle can also be applied for the home user with laundry machines and charging of e.g.
electric cars.
This practice would lead to reduced energy consumption and thus a reduced environmental footprint.
i3tex AB wants to investigate potential fields of applicability of this upcoming technology.
i3tex AB has customers in the automotive,
	communication,
	and pulp industries;
	those customers have made inquiries on how to integrate IoT and sensor networks into production.
As technology evolves,
	size and energy consumption of the IoT devices will decrease and computation power will increase [4].
This reduction in size and energy consumption,
	together with the increased computing power,
	will open up new fields for IoT.
Thus,
	i3tex AB want to have an IoT platform to present to their current and potential customers.
The interest in IoT is rapidly increasing,
	and thus,
	in the near future,
	the number of devices connected to the Internet is expected to increase rapidly.
To support this huge increase in both number of connected devices and the sheer amount of data that will be sent over both wired and wireless networks,
	the communication technology must be ready [5].
	
\subsection{Purpose (Goal)}

The purpose of this project is to find and examine a communication method for devices that are made to be a part of IoT.
This will be done by examining the available technologies and then developing a prototype based on the findings,
	which will be used for examining the communication method.
This project will examine the physical,
	link,
	and network layers [6, 7] of the Open Systems Interconnection model (OSI model) [8],
	in order to find suitable technologies on the market.
As IoT is still only defined as a concept,
	there are several technologies to take into consideration and examine in further detail.
The prototype will be delivered to i3tex AB together with appropriate documentation,
	e.g.
technical specification,
	hardware manual,
	software manual,
	and API specification.

\subsection{Limitations}

To be able to achieve the project goal within the available time,
	limitations need to be defined in the three main areas of:
	Operating System (OS),
	hardware,
	and communication method.
The OS will not be custom-made,
	but rather selected amongst those already on the market.
Thus,
	to simplify the hardware selection,
	only those OSs which already have hardware support that meets the requirements will be taken into consideration.
Furthermore,
	support for either 6LoWPAN,
	ZigBee,
	or Bluetooth Low Energy (BLE) as communication method is required,
	since development to make those standards available is outside the scope of the project.
On the hardware side,
	the limitations will be to only use existing devices and parts as there will be no time for developing hardware or Printed Circuit Boards (PCBs).
However,
	the hardware does not need to have an integrated radio transceiver,
	but needs to support at least one transceiver supporting IEEE802.15.4 [6].
Thus,
	communication methods will be primarily selected from specifications building on the IEEE 802.15.4.

\subsection{Method}

To ensure that the right technologies were selected and investigated,
	the first phase of the project was a literature study.
The study served as a foundation when developing and performing the evaluation of the communication methods.
At the end of the phase,
	a requirements specification was formulated to serve as a platform for the next phase.
After the literature study,
	a selection process was performed,
	where the most promising technologies that met the requirements were examined in further detail and brought into the development phase.
This process included the selection of development tools and other decisions bound to the product development.
In the development phase,
	the chosen set-up was configured and assembled to prepare for testing;
	it was then tested according to throughput,
	range,
	latency,
	and energy consumption.
Throughput was measured in kilobyte per second (KB/s) and tested by transferring data of different sizes in both congested and uncongested network set-ups to simulate real world and lab environments.
The same set-up was used to measure the latency of a transmission,
	which was measured in microseconds (ms).
Range was calculated instead of measured,
	with meters (m) as the unit.
The power consumption was measured in watts (W).
Each week a meeting with the company supervisors was performed,
	to keep the work on the right track.
Here,
	feedback was be given and other issues and questions handled.





\section{Results} \label{sec:Results}

Collecting the data went well and were reasonably straight forward;
	it was easy to transition between the two different test set-ups and thus making several test scenarios.
Assessments were made in the areas of range,
	response time,
	connection speed,
	and power consumption.
In each area,
	the theoretical values were first calculated and then compared to the retrieved measurements;
	except in the range case,
	as the required equipment for measuring was not economically justifiable to purchase.
	
\subsection{Range}

The theoretical range for OpenMote when transmitting at full power in an office environment is only 7m.
As measuring the range was not a viable option due to the cost of measuring equipment,
	only distance estimations from the placement of the nodes when maintaining a stable connection can be used as a reference.
Using a map of the office and the position of the nodes the range seems to be around 10m,
	which would mean that the effective FM of the office is around 16dB using the always-on RDC.
The FM changed a bit when using the 8Hz RDC as more packages congested the air and the range dropped to somewhere around 5m;
	resulting in an effective FM of ∼23dB.
To increase the range of the transceiver,
	a switch to the 860MHz frequency band would be the most effective solution;
	with a FM of 23dB,
	the theoretical range would increase to 14m with the same transceiver properties,
	and with a FM of 16dB the range would be 31m.
Usually,
	transceivers with a lower frequency output also have a lower power consumption while transmitting.
Working in sub-GHz also gives the benefit of less interference as fewer other devices uses those frequencies.
Changing to a sub-GHz band would thus decrease the power consumption and increase the range,
	without changing the functionality of the nodes.

\subsection{Response time}

Initially when measuring the response time the always-on RDC was used and the measured response time was very close to the theoretical value.
However,
	when using the 8Hz RDC protocol the values started to drastically differ from the theory.
This behaviour is likely to originate from the way the RDC driver predicts the next time when the target node should be awake.
The procedure is called phase optimization;
	when enabled,
	the node saves the time when the node was last seen,
	it then uses this value to predict the next time the node should be awake based on the RDC cycle.
However,
	this prediction is based on the node’s internal clock.
As the clock can differ from those of the other nodes,
	misalignments seem to occur,
	resulting in misses when trying to reach the target node.
Each misalignment increases the time it takes to reach the target node as the node then needs to strobe the package until the target nodes wakes up again.
In theory,
	when sending strobes the target node should wake up and receive the package within one cycle (125ms);
	however,
	this is not guaranteed as other transmissions might occupy the air,
	further increasing the response time.
If the phase optimization could be improved to guarantee the alignment between the nodes,
	the response time should get much closer to the theoretical value;
	as the time to reach the node would be maximum one cycle and the air would not be as congested by nodes sending strobes.

\subsection{Connection speed}

The connection speed,
	when using CoAP or any other protocol with perpacket ACK,
	is directly bound to the response time.
IEEE 802.15.4 has a relatively low data-rate,
	only 250kbps,
	compared to other solutions,
	e.g.
BLE (1Mbps) and WLAN (>54Mbps).
As throughput is based on datarate over a longer period of time,
	both the overhead and the response time is needed to make a good estimation.
CoAP has a very low header size compared to many other communication protocols,
	but due to the very small frame size,
	the overhead is still relatively high.
As of now,
	the results clearly show that when a reliable transfer is desired the connection speed of IEEE 802.15.4 and CoAP is only sufficient for data exchanges around 32 bytes.
When the nodes use the always-on RDC,
	the goodput is less than 3KB/s for one hop and is halved for every hop;
	however,
	when the 8Hz RDC is enabled,
	the goodput is reduced to under 0.1KB/s.
Using messages without per-packet ACK,
	thus removing the response time from the equation,
	would let the nodes transfer real-time audio and maybe even highly compressed video.
However,
	using messages without the per-packet ACK disables the reliable transmission guarantee,
	and thus it can only be used with data streams where packet loss is acceptable.


\subsection{Power consumption}

Making a rough estimation of the power consumption of the platform was straight forward task and so was measuring the actual consumption.
When comparing,
	the two the values differed by a factor of 30,
	which was not expected.
The reason probably originates from the clock interrupt which is triggered every 8ms.
Initially,
	this interrupt was assumed to be disabled when the system entered the lower power modes,
	but this was not the case.
As the interrupt fires at 125Hz and the time to wake up and go back to LPM is only 272μs,
	the power spikes from these interrupts were not seen on the measuring instruments.
As seen in figure 4.9,
	even the peaks from the listening cycles were hard to record and those lasted for at least 4ms;
	instead,
	the power consumption from the clock timer looks like an increased LPM power consumption.
At the time this was discovered there was no time to fix it,
	but doing so should decrease the average power consumption to within the limits,
	granting the nodes the ability to run on battery power for a year.
As no delays from calculation could be observed,
	the clock speed on MCU could,
	in all probability,
	have been reduced to save power on the nodes.
However,
	this reduction would only have affected the consumption when the node was in active mode,
	which is only a few percent of the total cycle time.
The OpenMote chip has a step-down DC-DC converter for this purpose which is switched off in LPM mode to reduce quiescent currents;
	however,
	as most of the time is spent in LPM,
	reducing the input voltage to 2.1V by changing battery type and removing the step-down converter would be preferable as it would reduce the power consumption.
These changes could affect the range of the device,
	but this has to be assessed.

\subsection{Project execution}

Looking at the time plan and the milestones,
	as seen in Appendix A and B,
	each milestone matches a task or transition in the time plan.
The planning report was not submitted to the examiner until the 6/2-15,
	which is two weeks behind schedule,
	exceeding the time planned for milestone M1.
The first draft was submitted before deadline,
	but several revisions were necessary.
In retrospect,
	the litterature study should probably have been planned in parallel with the planning report,
	as the information from the study helped with the report.
Milestone M2 marks the switch from the literature study and selection of technology to the development phase.
This milestone was met and development could begin in the following week.
As seen in Appendix C,
	the development phase have several risks to consider.
The only risk encountered in this phase was R4,
	as one of the hardware platforms was delivered with a broken sensor.
However,
	this malfunction did not affect the time plan as the development could continue regardless of the malfunction.
The end of the development phase was defined by milestone M3,
	approval of prototype,
	which was completed ahead of schedule granting an early transition into the assessment phase.
In the assessment phase,
	it could be argued that risk R9 was encountered when measuring the power consumption,
	as the results from those measurements did not properly show the wake-ups from the clock timer.
This phase contained milestone M4 and M5,
	of which of only M5 was done in time.
The Half-time presentation,
	milestone M4,
	was performed on the 8/4-15 in the form of a meeting,
	where the progress,
	results and continuation plan were discussed.
Also,
	a half-time version of the report was sent the 17/4-15 and approved by the examiner.
Milestone M6,
	deliver the final prototype,
	was completed a few days before the set deadline which eliminated risk R11 and gave more time to work on the writing and the presentation.
Both of the oral presentations were attended on the 1/6-15 to grant some experience in how the presentation and opposition are carried out,
	thus now following the time plan.
However,
	there were not many presentations to watch during the planned weeks,
	as the presentation schedule follow the academic semesters.
The presentation for this thesis was not performed until the 3/6-15,
	thus being two weeks behind schedule.
However,
	it was scheduled on the first available date suggested by the institution.
The final version of the report will be submitted to the examiner before the 19/6-15,
	thus successfully completing milestone M7.














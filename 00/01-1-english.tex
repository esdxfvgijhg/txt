Nowadays,
	Internet of things is witnessing a tremendous evolution due to the increasing growth in communication technologies,
	weather and environmental sensing, health care sensing.
Indeed,
	sensors are being a kind of intelligent mobile agent able to perceive its environment and transmit information to the cloud for processing,
This way of perception allow the development of several kinds of applications to enhance human capacity to understand their environment and make appropriate decision.
However,
	developing such advanced applications relies heavily on the quality of the communication between sensors and between sensors and the infrastructure,
	therefore,
	such communication can be realized only with the help of a secure data collection and efficient data treatment and analysis.

%Thematique
Data collection in a vehicular network has been always a real challenge due to the specific characteristics of these highly dynamic networks (frequent changing topology,
	vehicles speed and frequent fragmentation),
	which lead to opportunistic and non long-lasting communications.
Security,
	remains another weak aspect in these wireless networks since they are by nature vulnerable to various kinds of attacks aiming to falsify collected data and affect their integrity.
Furthermore,
	collected data are not understandable by themselves and could not be interpreted and understood if directly shown to a driver or sent to other nodes in the network.
They should be treated and analyzed to extract meaningful features and information to develop reliable applications.
In addition,
	developed applications always have different requirements regarding quality of service (QoS).
Several research investigations and projects have been conducted to overcome the aforementioned challenges.
However,
	they still did not meet perfection and suffer from some weaknesses.
For this reason,
	we focus our efforts during this thesis to develop a platform for a secure and efficient data collection and exploitation to provide vehicular network users with efficient applications to ease their travel with protected and available connectivity.
Therefore,
	we first propose a solution to deploy an optimized number of data harvesters to collect data from an urban area.
Then,
	we propose a new secure intersection based routing protocol to relay data to a destination in a secure manner based on a monitoring architecture able to detect and evict malicious vehicles.
This protocol is after that enhanced with a new intrusion detection and prevention mechanism to decrease the vulnerability window and detect attackers before they persist their attacks using Kalman filter.
In a second part of this thesis,
	we concentrate on the exploitation of collected data by developing an application able to calculate the most economic itinerary in a refined manner for drivers and fleet management companies.
This solution is based on some information that may affect fuel consumption,
	which are provided by vehicles and other sources in Internet accessible via specific APIs,
	and targets to economize money and time.
Finally,
	a spatio-temporal mechanism allowing to choose the best available communication medium is developed.
This latter is based on fuzzy logic to assess a smooth and seamless handover,
	and considers collected information from the network,
	users and applications to preserve high quality of service.




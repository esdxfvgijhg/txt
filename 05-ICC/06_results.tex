\section{Results exploitation} \label{sec:Results exploitation}

This section shows some simulation results and compares the performance of our proposed solution to three different reference \ac{ADR} algorithms.

\subsection{servicetype-based selection algorithm}
The first algorithm is a servicetype-based selection algorithm where high bit services with low propagation delay requirements are sent to the WLAN and the low bit rate services with the high propagation delay requirements are sent to the WWAN.

\subsection{terminal speed-based selection algorithm}
The second algorithm is a terminal speed-based selection algorithm where highspeed users are sent to the large-coverage network (i.e.,
	WWAN) and the low-speed users are sent to the smallcoverage network (i.e.,
	WLAN).

\subsection{random-based selection algorith}
The third algorithm is a random-based selection algorithm where the users are assigned randomly to the two networks.

All solutions have been simulated,
	evaluated,
	and compared for the three different objectives addressed in Section x.
Several runs of simulation have been carried out.
Each run uses a different number of users.


\subsection*{Results interpretation}
% Figure
Figure x shows $P_{u}$ values in all solutions.
The horizontal axis shows the number of users while the vertical axis shows the $P_{u}$ values.

% Figure
From both Figure x and the numerical samples for $P_{u}$ values shown in Table x,
	the great improvement in the number of the satisfied users in our solution can be seen.
For example,
	with 840 users in the environment,
	the percentage of satisfied users with all different reference solutions is around 50\% while the same percentage in
our proposed solution is around 80.7\%.
In general,
our proposed solution achieves around 31\% enhancement over the different reference algorithms.

% Figure
In Figure x,
	the lower plot shows the weights assigned by the GA for the different criteria using the first objective function.
The upper plot shows the best function value in each generation versus iteration number.

% Figure
Figure x shows $P_{q}$ values in all solutions.
The horizontal axis shows the number of users while the vertical axis shows the $P_{q}$ values.

% Figure
From both Figure x and the numerical samples for $P_{q}$ values shown in Table x,
	the great improvement in the percentage of the users assigned to networks with stronger received signal can be seen.
For example,
	with 840 users in the environment $P_{q}$ with the different reference algorithms is around 50\% while the same percentage in our proposed solution is around 64\%.
In average,
	our proposed solution achieves around 26.5\% enhancement over the random-based selection.

% Figure
Figure x shows the weights assigned by the GA for the second objective.
The results of $P_{o}$ produced by all solutions for the third objective are shown in Table x.

Our proposed solution achieves good and comparable results against the results achieved by other solutions.
Figure x shows the weights assigned by the GA for the third objective.



















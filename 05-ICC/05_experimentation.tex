\section{Experiments} \label{sec:Experimentation}

Our proposed scheme and SA are evaluated using the simulation approach.

MATLAB mathematical software and a set of functions called RUNE [1] have been used for the simulation.
The system model considers the coexistence of a CDMA-based WWAN network with seven macrocells with omnidirectional antenna and cell radius = 1000 m,
	and a TDMA-based WLAN network with twelve microcells with omnidirectional antenna and cell radius = 500 m.
In the system environment,
	each mobile has a velocity and is moved with a random distance and a random direction at defined time steps.
The velocity is a vector quantity with magnitude and direction.
The traffic is modeled according to Poisson process.
The main call holding time is assumed to be 50 seconds.
Four types of services are considered in our simulation,
\Itemize{
	\item he voice calls,
	\item the low bit rate real-time video telephony,
	\item the high bit rate streaming video,
	\item and the nonreal-time data traffic.
}


The velocity of the ith mobile is updated according to (4);
\Equation{tf}{G = GD + GF + GR + GA}

where:
\Itemize{
	\item $U_{i}$ is the complex speed [m/s].
	\item P is the correlation of the velocity between time steps.
	\item X is Rayleigh distributed magnitude with mean 1 and a random direction.
	\item $V_{m}$ is the mean speed of mobiles.
}

The propagation model used in this paper can be described in logarithmic scale as in (5)
\Equation{tf}{G = GD + GF + GR + GA}
where:
\Enumerate{
	\item The first component is the distance attenuation GD,
		which is given by the Okumura-Hata formula (x):
		where:
		\Itemize{
			\item d is the distance to the transmitter,
			\item β is a constant used to model the effect of carrier frequency,
				the antenna size and other physical parameters.
				It  has been set to − 28 dB
			\item The parameter α is the distance attenuation coefficient.
				It has been set to 3.5.
			} 
	\item The second component is the shadow fading GF,
		which is modeled as a log-normal distribution with standard deviation of 6 dB and 0 dB mean.
	\item The third component is the Rayleigh fading GR,
		which is a Rayleigh distributed.
	\item The fourth component is the antenna gain GA.
}



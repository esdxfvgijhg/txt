\section{Approach} \label{sec:Approach}


$\min _{W, Z} F(W, Z)=\sum_{l=1}^{k} \sum_{i=1}^{n} w_{l i}^{x} d\left(\mathbf{z}_{l}, \mathbf{x}_{i}\right)$
such that
$0 \leq w_{l i} \leq 1, \quad 1 \leq l \leq k, \quad 1 \leq i \leq n$
$\sum_{l=1}^{k} w_{l i}=1, \quad 1 \leq i \leq n$
$0<\sum_{i=1}^{n} w_{l i}<n, \quad 1 \leq l \leq k$


% d A
% n O
% k C

% ajr
% zlj
% dc
% kd
% $A_{(1<j<d)}$={$a_{j1}$, ..., $a_{jc}$$} a
% $Z_{(1<l<k)}$={$z_{l1}$, ..., $z_{ld}$}
% A1 Ad

where:
\Itemize{
	\item  n is the number of possible configurations $x_{i}$ with 1<i<n.
	\item  k is the number of application clusters $c_{l}$ with 1<l<k.

	\item  d is the number of metrics $a_{j}$ of each object $x_{i}$ with 1<j<d.
	\item  c is the number of metrics categories (physical, mac) $c_{m}$ with 1<m<c.

	\item  A={$A_{1}$, ..., $A_{d}$} is a set of categorical attributes with $A_{(1<j<d)}$={$a_{j1}$, ..., $a_{jc}$}

	\item  D={$x_{1}$, ..., $x_{n}$} is a set of n objects.
	\item  D={$x_{1}$, ..., $x_{n}$} categorical data set with $x_{(1<i<n)}$ \in $A_{(1<j<d)}$

	\item  $Z_{l}$={$z_{l1}$, ..., $z_{ld}$} is a set of cluster centers.

	\item  W=$w_{li}$ is a k x n fuzzy membership matrix.
	\item  \alpha \in[1, ] is a weighting exponent.
	\item  d($z_{l}$, $x_{i}$) is a certain distance measure between cluster center $z_{l}$ and the object $x_{i}$.
}

\subsection{The FL-based Control Component}
Our framework contains four FL-based subsystems.
Each subsystem considers one of the \ac{ADR} important criteria.


\subsubsection{Inputs}
\paragraph{Physical subsystem}
The Physical subsystem considers the received signal strength criteria.
It has two input variables RSS1 to describe the received signal strength from LoRa-based network,
 and RSS2 to describe the received signal strength from GFSK-based network.
\paragraph{Envirement subsystem}
The Envirement subsystem considers the mobile station speed criteria.
It has only one input variable MSS to describe the mobile station speed.
\paragraph{Sercice subsystem}
The Sercice subsystem considers the service-type criteria.
It has two input variables,
	the first is “DelayReqc” to describe the one-way delay needed for the required service,
	 and the second is “RateReqc” to describe the bit rate needed for the required service.
\paragraph{User subsystem}
The User subsystem considers the user preference and price criteria.
It has only one input variable “Price” to describe the user preferred price.
\paragraph*{membership functions}
Every input variable has three membership functions 
\Itemize{
	\item low
	\item medium
	\item high 
}
Figure 3 shows the membership functions of the input variables.

\subsubsection{Outputs}
Every subsystem has two output variables,
\paragraph{CDMA network}
one variable to describe the probability of acceptance for the new user in the CDMA network
\paragraph{CDMA network}
one variable to describe the probability of acceptance for the new user in the TDMA network.
The subsystems output variables are 
\Itemize{ 
	\item $RSS_{c1}$, $RSS_{c2}$ for RSS subsystem,
	\item $MSS_{c1}$, $MSS_{c2}$ for MSS subsystem,
	\item $ST_{c1}$, $ST_{c2}$ for ST subsystem,
	\item $UP_{c1}$, $UP_{c2}$ for UP subsystem.
}
\paragraph*{membership functions}
Each output variable has four membership functions 
\Itemize{ 
	\item TR (totally reject)
	\item PR (probability reject)
	\item PA (probability accept)
	\item TA (totally accept) 
}
Figure 4 shows “ST c1 ” variable with its membership functions as a sample for the output variables.

\subsection{The GA Component}

The GA has been used in our scheme to help the users or network operator to find suitable values for the weights {$W_{v}$,
	$W_{s}$,
	$W_{t}$,
	$W_{u}$}.


Beside the GA advantages that are summarized on Section 2.4,
	our decision to use GA was based on the nature of our objective functions that have several dynamic and stochastic components,
	where any other derivative-based optimization method cannot perform well.
Another important issue that encouraged the selection of GA to our problem is the high interaction between different variables.

Since the \ac{ADR} is a multi criteria problem in nature,
	different objectives need to be optimized.

In this paper,
	three different objective functions are proposed to cover the different and opposite objectives and requirements of the users,
	QoS and operators.
The proposed objective functions are outlined in the following points.


\subsubsection{Objective function 1 (UP)}
The first objective is to maximize the percentage of satisfied users,
	that is,
	users who are assigned to a network of their preference ($P_{u}$ ):
	since the algorithm considers multiple criteria,
	it may assign a user to a network different from his/her original preference.
$P_{u}$ is considered appropriate for performance evaluation from the user point of view.
To maximize $P_{u}$ ,
	the objective function ObjFun1 shown in Algorithm 1 is used.

\subsubsection{Objective function 2 (RSS)}
The second objective is to maximize the percentage of users assigned to the networks with stronger signal strength ($P_{q}$):
	since the algorithm considers multiple criteria,
	it may assign a user to a network with weaker signal strength.
$P_{q}$ is considered as a simple indicator for performance evaluation from the QoS point of view.
To maximize $P_{q}$ ,
	the objective function ObjFun2 shown in Algorithm 2 is used.

\subsubsection{Objective function 3}
The third objective is to achieve the load balancing between networks:
	P o is the percentage between the number of users in both networks.
It is used as a basic indicator for load balancing.
If the resources of both networks have the same importance and cost for network operator,
	this objective is considered appropriate for performance evaluation from network operator point of view.
To achieve load balancing,
	the objective function ObjFun3 shown in Algorithm 3 is used.



The weights of the input criteria {$W_{v}$,
	$W_{s}$,
	$W_{t}$,
	$W_{u}$} have been encoded using real encoding method.
The length of the real-valued encoding is 4 real floating numbers.
The length of each floating number is depending on the internal precision and roundoff used by the computer to define the precision of the floating numbers.
The results achieved by extensive comparisons of GA performance as a function of the different GA parameters (i.e.,
	the population size,
	the mutation rate,
	and the crossover fraction) and the different GA operators (i.e.,
	the selection and crossover operators) have been summarized by [27].


They have concluded that crossover fraction,
	selection operator,
	and crossover operator are not of much importance.
On the other side,
	the population size and the mutation rate have the most significant impact on the ability of the GA to find better minimum value for objective function.
Based on that,
	several sets of experiments to determine good population size and mutation rate for our proposed objective functions have been carried out.
Each experiment has been repeated several iterations while the performance is recorded.
The results from all the iterations are then combined by calculating the average (and standard deviation) for each experiment.
Usually,
	testing GAs includes mainly two issues;
	how far the result obtained by GA is from the benchmark results,
	which can be measured by average fitness value.
The second issue is how fast GA is in finding the best solution,
	which can be measured by the total number of function evaluations.

\paragraph{Parameter initialization}
Since our GA is working offline,
	only the average best fitness value achieved by the GA is used.
The average best fitness and the standard deviation are plotted against the tested parameters.
When studying a parameter all the other parameters and operators are kept constant.
Only a sample for the experiments results achieved is given in this paper as shown in Figure 5.

The lower plot in Figure 5 shows the me\ac{ADR} and standard deviations of the ObjFun1 function best fitness values over 20 runs,
	for each of the values of the population sizes between 10 and 100 with step size equals to 10 individuals.

The upper plot shows a color-coded display of the best fitness values in each run where dark color indicates better results.
For ObjFun1 fitness function,
	setting the population size to 20 individuals or more yields better result.
Based on these results and the other experiments results for other objective functions,
	the population size chosen in our GAs has been set to 20.

Other sets of experiments have been carried out to determine a suitable mutation rate.
Based on the experiments,
	the mutation rate has been set to 0.1.
In addition,
	our GA uses a roulette selection function.

\paragraph{Elitism}
To ensure elitism,
	the number of individuals that are guaranteed to survive to the next generation has been set to 4 individuals.
It uses two-point crossover function and a uniform mutation function.
The GA terminates if the maximum number of the GA iterations reaches to 300 iterations,
	if there is no improvement in the best fitness value for number of consecutive generations equals to 100 generation,
	or if there is no improvement in the best fitness value for an interval of time equals to 300 seconds.


\subsection{The MCDM Component}

The input criteria of the MCDM are the outputs of the FL-based control subsystems in the first component.
The GA component assigns the weight w i for criteria i to reflect its relative importance.
The criteria with more importance to the operator and user can be assigned higher weight using the objective function of the GA specified by the operator.
Since all the outputs of FL subsystems are in the range [0, 1],
	there is not be any need to scale the criteria performance against alternatives.

\paragraph{simple multi-attribute rating technique (SMART)}

Enhanced version of simple multi-attribute rating technique (SMART) has been used.
SMART is one of the simplest MCDM and most efficient methods.
The ranking value x j of alternative A j is obtained simply as the weighted algebraic mean of the utility values associated with it,
	that is,
	a i j according to (1):

SMART employs relatively uncomplicated and straightforward manipulation method,
	which makes it stronger and easier to use in a hybrid and more complex models such as the proposed one in this paper.
With the aid of both FL and GA,
	SMART has all the capabilities required to address the specific considerations that are involved in the \ac{ADR} decision making process.
SMART can be quickly and easily understood by the inexperienced decision makers.
However,
	to get better understanding and deeper insights into the selection decision making,
	other advanced MCDM methods can be used in the future work and compared with SMART method.
In our algorithm,
	there are two alternatives for the MCDM,
	one is a CDMA-based network and the other is a TDMA-based network.

\paragraph*{Ranking function}
The ranking value of CDMA network x c and the ranking value of TDMA network x t can be calculated as follows:

where:
\Itemize{
	\item $W_{v}$ is the assigned weight for the MSS subsystem.
	\item $W_{s}$ is the assigned weight for the RSS subsystem.
	\item $W_{t}$ is the assigned weight for the ST subsystem.
	\item $W_{u}$ is the assigned weight for the user UP subsystem.
	\item $T_{w}$ is the total weight and is calculated using (3):
}

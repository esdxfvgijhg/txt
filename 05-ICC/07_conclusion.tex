\section{Conclusion} \label{sec:Conclusion}



A novel framework to solve the \ac{ADR} problem has been presented in this paper.

The proposed framework is scalable and is able to handle the huge number of configurations with a large set of criteria.

The framework can cope with the different and conflict view points and purposes of the operator and users,
  it can react to the accumulated end users experience about the \ac{ADR}.



The proposed framework has been developed to present and design a multi-criteria \ac{ADR} solution that considered the end user,
	the application QoS requirements,
  the environment conditions,
	and the operator view points.


The main focus of our work is to use a multi-objective optimization method to find the optimum weights jointly,
  rather than using the single-objective GA that optimize each weight independently.

As the proposed membership functions and rules of the fuzzy subsystems are still subjective,
	they can be tuned using a suitable learning method such as the neural networks or the genetic algorithms.


%Different aspects of our work have to be developed.
%More criteria can be included such as signal-to-noise ratio (SNR) and resources availability.





%% Restate the main challenges ------------------------------------------------
%The main challenge of this work was to
%The efficiency of such algorithms

%% Restate the main contribution ------------------------------------------------
%Our main contribution was to

%% Restate the main findings --------------------------------------------------
%To measure the accuracy of

%% Future challenges current bad state -------------------------------------------
%As we find that ..., we plan to  ..





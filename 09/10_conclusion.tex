\begin{frame}{Conclusion} % challenges overcome ? not really ? 90%


\Itemize{
	\item The main challenge of this work was to build a tool that can \textbf{easily be plugged} in \acs{LoRaWAN} network servers \textbf{to} map transmission settings to applications requirements.
	\item Our contribution was to test the effectiveness of applying the \textbf{\acs{FCM} clustering algorithm} to select the transmission setting that best fit a given application requirement.
	\item Each cluster \textbf{represents} a set of applications with the same \ac{QoS} requirements.
	\item The proposed process has been developed to present and design a solution that consider \textbf{radio parameters} (\ac{SF},\ac{BW} and \ac{PS}),
	\textbf{environment conditions} (\ac{SNR}) and performance metrics (\ac{ToA}, \ac{BER} and \ac{RSSI}) required by applications.
	\item We plan to \textbf{integrate this approach} in one of the open source \ac{LoRaWAN} network servers like the ChirpStack network server to test their performance in a real environment.

}

% %The main issue of our work is how to select LoRa network settings adapted to QoS requirements? We proposed FCM clustering to classify the settings to three different clusters based on three QoS metrics.
% Simulation results have shown that the \ac{FCM} clustering algorithm is efficient and able to cluster all possible settings to the expected three clusters.
% Furthermore,
% 	after the application of \ac{FCM} algorithm,
% 	the settings have been ranked based on their membershipness.
% The proposed process has been developed to present and design a solution that consider \ac{LoRa} parameters (\ac{SF},\ac{BW} and \ac{PS}),
% 	environment conditions (\ac{SNR}) and performance metrics (\ac{ToA}, \ac{BER} and \ac{RSSI}) required by applications.
% We plan to integrate this approach in one of the open source \ac{LoRaWAN} network servers like the ChirpStack network server to test their performance in a real environment.
% This could allow the network server to select the best configuration with a higher membership value to applications running on end-devices.

	% \begin{table}[h!]
	% 	\begin{tabular}{l|l|l|l|l|l}
	% 	\textbf{Contributions}    & Memory	& Computation	& Dynamic 	& Optimality 	& Costs		\\\hline
	% 	Contribution 1            &  \ok	    &  	\ko			&    \ko		&  	\ko		 	&	\ok		\\\hline
	% 	Contribution 1            &   \ko  	& 		\ko		&	 \ko   	& 		 \ok		&		\ko	\\\hline
	% 	Contribution 1            &  	\ok    & 	\ko			&    \ok		& 	\ko			&	\ko		\\\hline
	% 	Contribution 1            &  	\ok	& 				&    \ko		& 	 \ko			&	\ko		\\\hline
	% 	Contribution 1            &  	\ok	&  		\ok		& 	 \ok	  	& 		\ok		&		\ok	\\\hline
	% 	\end{tabular}
	% \caption{\label{tab:} }
	% \end{table}





  \bey{blue}{0}{.5\columnwidth, 0}{Thank you !}
  %\stamp{blue}{30}{6.2, 5}{How to adapt network configurations to applications ?}{90}

\end{frame}


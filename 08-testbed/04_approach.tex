\section{Proposed ...} \label{sec:Approach}

The goal of the development process is to have a functional border router and at least two nodes to be able to test how response time and throughput differs with each hop in a mesh network.
To be able to measure response time and throughput,
	each node needs to have a CoAP server which can respond to ping and also receive an arbitrary amount of data for throughput measurement.
It is desirable for each node to be able to send information about each sensor so the project can be used as a tech-demo.
The first part of the development was to set-up of the workspace and tools mentioned in section 3.1.5.
Ubuntu OS was installed in a VirtualBox Virtual Machine to make it easier to duplicate and backup;
	this procedure gave a noticeable decrease in performance and it is recommended to have a dedicated native Ubuntu machine for this type of development.
Even though Ubuntu uses an easy-to-use package system,
	there were some problems in finding a version of GCC ARM Embedded tool-chain that was compatible with Contiki’s built-in simulator Cooja [34, 35];
	eventually,
	version 4.82 was used to successfully build Contiki.
Cooja is a useful tool for testing and debugging network configurations but does not have support for the CC2538 MCU;
	instead,
	nodes called Cooja Motes are simulated with generic hardware.
As Cooja is written in Java and runs in a JVM,
	Oracle Java 1.8 was also installed.
	
\subsection{Drivers and firmware}

Figure 3.3 shows an overview of the full system.
The foundation is the SoC with the MCU,
	transceiver,
	and sensors.
The Contiki operating system implements the soft real-time kernel together with the firmware for the SoC/MCU and the drivers for the peripherals and sensors.
The last part is the communication stack,
	which provides TCP and UDP connectivity over 6LoWPAN.
On top of the TCP and UDP protocols,
	HTTP and/or CoAP can be implemented.
The firmware required for the OS to work properly on the hardware platform was already implemented.
However,
	the drivers for the I 2 C bus and the sensors were not implemented.
The I 2 C driver is required for the sensor drivers which in turn enables the MCU to communicate with the sensors on the OpenBattery platform.

\subsection{CoAP server}

In order to make each node’s sensor data accessible,
	a CoAP sever was implemented as an application running on top of Contiki.
A CoAP server in general can handle any number of resources;
	in this implementation,
	one resource was made for each sensor value i.e.
temperature,
	light,
	humidity,
	and core voltage.
The temperature,
	light,
	and humidity sensors all work in a similar fashion.
When their value is requested,
	the I 2 C bus is initialized and then a request is sent over the bus.
When the response with the value arrives,
	that data is put into either a plain-text or JavaScript Object Notation (JSON) formatted message depending on the request and then sent back to the requester.
As the core voltage sensor is part of the MCU’s ADC,
	that value is retrieved by simply getting data from a register (a somewhat faster operation).
As a buffer for testing throughput speed was also needed,
	a resource with a circular buffer was implemented.
This resource is configured with CoAP’s block-wise transfer functionality for arbitrary data size;
	however,
	the buffer in itself is only 1024Kb to allow the program variables to fit into the ultra low leakage SRAM.
For testing purposes,
	the data could have been discarded instead of actually saved into the buffer,
	but then the transfer can not be verified.
Resources are defined by paths as CoAP works in a very similar way as HTTP.

Each resource is registered in the server with its path,
	media type,
	and content type.
When a package arrives on the CoAP port,
	the server starts to break down the package to be able to direct it to the right resource.
It starts with verifying that the package is actually a CoAP package,
	and then it checks the path and sends it to the correct resource.
The resource then inspects the method field in the package header to direct the incoming data to the right function.
CoAP package method can be either GET,
	PUT,
	POST or DELETE.
This function then inspects the request media type and answer content type so that the function can parse the request and send a correctly formatted answer.
If the resource does not implement the received method,
	the server responds with ”405 Method not allowed” and if the content/media type is not supported the answer is ”415 Unsupported Media Type”.
The content/media types are text/plain,
	application/json,
	application/exi,
	and application/xml.

\subsubsection{Testing}
Contiki is shipped with a simulation tool called Cooja which is written in Java;
	it can simulate an arbitrary number of nodes with different roles and configurations.
All simulation data,
	such as radio packages and node serial output may be viewed through different windows and exported to various formats.
Unfortunately Cooja did not have support for ARM Cortex-M3,
	but the general set-up was still tested by using Cooja Motes,
	which are nodes without specified hardware,
	and MSP430 nodes such as Wismote or Skymotes.
With this simulator the basic understanding of the communication between nodes was gained;
	also,
	before the hardware arrived,
	early testing was performed to test the OS and application software.

\subsubsection{Final prototype}
The final prototype consists of four OpenBattery nodes and one OpenBase border router.
Both the nodes and the border router are deployed with Contiki.
Each node runs a CoAP server,
	described in section 3.2.2,
	on top of the OS in its own thread.
The border router runs a router software called 6lbr that acts a as translator between Ethernet and IEEE 802.15.4 [36].
Both types of hardware are configured with a 8Hz Radio Duty Cycle (RDC) driver to keep the power consumption to a minimum.
RDC is a OS driver that cycles the listening mode of the transceiver to reduce power consumption.
As Contiki puts the MCU into Low Power Mode (LPM) when no function is running and the transceiver is off,
	the RDC driver indirectly controls when the MCU is in LPM.
When using the RDC protocol,
	the nodes repeatedly send messages until the target node wakes up and sends an Acknowledge packet (ACK);
	this makes communication seamless,
	even though most of the time the nodes’ transceivers are not active.
Also,
	an always-on RDC driver,
	where the transceiver is constantly listening,
	will be used to be able to look at the performance impact of the 8Hz RDC.






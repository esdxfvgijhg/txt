\section{Related work} \label{Sec:Related_Works}

Petri nets (PNs) are widely used for traffic light modelling and control \cite{huang_modular_2014}.
In  \cite{difebbraro_trafficresponsive_2006},
	deterministic-timed Petri Nets have been used to describe signalized intersections.
Undesirable deadlock states might appear when the nets are tested for some use cases.
The authors in \cite{febbraro_using_2009} have modified PNs models including  stochastic-time for one single signalized intersection.
Dotoli and Fanti \cite{dotoli_urban_2004} have built a colored timed PN with a deterministic modular framework,
	in which parts of the system,
	and even parts of the subsystems,
	can be specified and analyzed separately.
Examples using modularity are given in Soares and Vrancken \cite{dossantossoares_modular_2012},
	in which a p-timed PN is used for the control of a traffic signal in both main road and side streets.
However,
	formal characteristics of PNs (\emph{e.g.},
	deadlock and liveliness) haven’t been discussed.
Moreover,
	PNs suffer from a lack of analysis and verification tools.
To overcome these limits of PNs,
	we propose UPPAAL timed automata for design and verification of coherent state of cross road's traffic light.
UPPAAL is a timed-based modelling software with a graphical user interface.
It is the result of the research works of two universities UPPsala University in Sweden (UPP) and AALborg University in Denmark (AAL) \cite{david_uppaal_2015}.

In \cite{Web0},
	thermal cameras and on-street wired sensors detect vehicles and pedestrians in order to adapt the cycle of traffic light control systems.
However,
	such a solution can be expensive.
In addition,
	the system uses only its local view of the environment to detect the arrival of a vehicle.
Other solutions use recent technologies such as wireless sensors devices to limit the cost of thermal cameras and reduce the time needed to deploy sensors.
In \cite{tlig_decentralized_2014} and \cite{rose_internet_2015},
	the authors propose an adaptive system based on local wireless communication between lights and vehicles.
But such a solution requires a global interconnection between all road's users and infrastructure.
This problem comes from the rigid definition of technologies' standards.
Our work is not only limited to establish WSN,
	but it is scalable to interconnect heterogeneous wireless technologies through the Internet.
The obtained WSN intends to meet multiple QoS requirements of IoT applying the MQTT protocol.
In \cite{Silva2018},
	the latency of MQTT has been evaluated by calculating the average round-trip delay between two clients located in two different continents.
However,
	the evaluation has been limited to the impact of messages' size.
In our work,
	we consider the period of generated messages,
	and we calculate the RTT delays from WSN to Cloud IoT plateform.

In \cite{huang_modular_2014} \cite{difebbraro_trafficresponsive_2006} \cite{febbraro_using_2009} \cite{dossantossoares_modular_2012},
	the authors focus only on the structural analysis of their models and the transitions between colors of traffic lights.
However,
	the implementation of their models as a service in the Internet of smart cities has not been discussed.
Moreover,
	their methodology is not tested with any real traffic lights' Testbed.

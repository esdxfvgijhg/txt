\begin{abstract}

% Statistics
The exponential usage of messaging services for communication raises many questions in privacy fields.
% Problem
Privacy issues in such services are strongly related to the graph-theoretical properties of users' interactions representing the real friendships between users.
One of the most important issues of privacy is that users may disclose information of other users beyond the scope of their interaction,
	without realizing that such information could be aggregated to reveal sensitive information.
% Challenges
Determining vulnerable interactions from non-vulnerable ones is difficult due to the lack of awareness mechanisms.

% Contribution
To address this problem, we analyze the topological trust relationships between users to notify each of them about their vulnerable social interactions.
% Contribution Specifically, Particularly
Particularly,
	we analyze the impact of trusting vulnerable friends in affecting other users' privacy concerns by modeling a new vulnerability diffusion process.
% Experimentation & results
Simulation results show that over-trusting vulnerable users speeds the vulnerability diffusion process through the network.
Furthermore,
	vulnerable users with high reputation level spread their vulnerability widely trough the network,
	this means that the vulnerability diffusion process affects the biggest number of users when vulnerable users get a high level of trust from their interlocutors.
% Future work
This work contributes to the development of privacy awareness framework that can alert users of the potential private information leakages in their communications.

\end{abstract}


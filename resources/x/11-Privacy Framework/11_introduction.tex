\subsection*{Statistiques}
% Actual situation

\begin{frame}{Context}{Introduction}
	
	\Table{|l|l|l|l|}{emailStat}{Les chiffres 2018 de l’émail \cite{BibEntry2014Sep}}{
		\                                          & Monde (2018)        & Monde (2022)        & France (2018) \\\hline
		\ Nombre d’utilisateurs                    & 3,8 milliards       & 4,2 milliards       & 25,9 millions \\
		\ Nombre de comptes émail                  & 4,4 milliards       & 5,6 milliards       & 68 millions   \\
		\ Nombre d’adresses émail par utilisateurs & 1,7                 & 1,9                 & 2,1           \\
		\ Nombre de mails reçus chaque jour        & \red{281 milliards} & \red{333 milliards} & \red{1,4 milliard}  \\
		\ Le marché de l’émail                     & 9,8 Mrds de \$      & 20,4 Mrds           & ?             \\\hline
	}

	\Columns{0.6}{0.4}{
		\Itemize{
			\item Email:
			\Itemize{
				\item 50\% de la population mondiale utilise l'émail
				\item Utilisation: 75\%  personnels, 25\% professionnels
			}
			\item Facebook:
			\Itemize{
				\item 2,2 milliards d’utilisateurs actifs (29\% mondiale)
				\item \red{10 milliards de messages sont envoyés chaque jour}
				\item 8,051 milliards de dollars
			}
		}
	}{
		\Figure{p}{.5}{Messaging.png}{Services de messagerie}
	}

\end{frame}


\begin{frame}[noframenumbering]{Context}{Introduction}
	
	\TableT{|l|l|l|l|}{emailStat}{Les chiffres 2018 de l’émail \cite{BibEntry2014Sep}}{
		\                                          & Monde (2018)         & Monde (2022)    & France (2018) \\\hline
		\ Nombre d’utilisateurs                    & 3,8 milliards        & 4,2 milliards   & 25,9 millions \\
		\ Nombre de comptes émail                  & 4,4 milliards        & 5,6 milliards   & 68 millions   \\
		\ Nombre d’adresses émail par utilisateurs & 1,7                  & 1,9             & 2,1           \\
		\ Nombre de mails reçus chaque jour        & 281 milliards        & 333 milliards   & 1,4 milliard  \\
		\ Le marché de l’émail                     & \red{9,8 Mrds de \$} & \red{20,4 Mrds} & \red{?}             \\\hline
	}
	
	\Columns{0.6}{0.4}{
		\Itemize{
			\item Email:
			\Itemize{
				\item 50\% de la population mondiale utilise l'émail 
				\item Utilisation: 75\%  personnels, 25\% professionnels
			}
			\item Facebook:
			\Itemize{
				\item 2,2 milliards d’utilisateurs actifs (29\% mondiale)
				\item 10 milliards de messages sont envoyés chaque jour
				\item \red{8,051 milliards de dollars}
			}
		}
	}{
		\FigureS{p}{.5}{Messaging.png}{Services de messagerie}
%		\Figure{p}{.76}{or.jpg}{Services de messagerie}
	}

\end{frame}




\subsection*{Problématique}
% What's the problem

\begin{frame}{Problématique}{Introduction}

	\Columns{0.65}{0.35}{
		\Enumerate{
			\item Vulnérabilités externes
			\Itemize{
				\item Les technologies réseaux: 4G, 5G, Wifi, Ethernet.
				\item Les protocoles de sécurité: HTTPS, SMTPS, DNS, IPsec
			}
			\item Vulnérabilités internes
			\Itemize{
				\item Fournisseurs des services
				\Itemize{
					\item Conditions générales d'utilisation (CGU)
				}
				\item 3$^{rd}$ parties
				\Itemize{
					\item Configuration de la confidentialité
					\item Gestion des permissions
				}
				\item Utilisateurs
				\Itemize{
					\item Configuration des comptes utilisateurs
					\item Gestion des listes de contactes
				}
			}
		}
	}{
		\towFigure{0.8}{Privacy.jpg}{Privacy2.jpg}{privacy}{Confidentialité}
	}
\end{frame}

\begin{frame}[noframenumbering]{Problématique}{Introduction}

	\Columns{0.65}{0.35}{
		\Enumerate{
			\item Vulnérabilités externes
			\Itemize{
				\item Les technologies réseaux: 4G, 5G, Wifi, Ethernet.
				\item Les protocoles de sécurité: HTTPS, SMTPS, DNS, IPsec
			}
			\item Vulnérabilités internes
			\Itemize{
				\item Fournisseurs des services
				\Itemize{
					\item Conditions générales d'utilisation (CGU)
				}
				\item 3$^{rd}$ parties
				\Itemize{
					\item Configuration de la confidentialité
					\item Gestion des permissions
				}
				\item \blue{Utilisateurs}
				\Itemize{
					\item \blue{Configuration des comptes utilisateurs}
					\item \blue{Gestion des listes de contactes}
				}
			}
		}
	}{
		\towFigureT{0.8}{Privacy.jpg}{Privacy2.jpg}{privacy}{Confidentialité}
	}
\end{frame}


\subsection*{Motivation}
% Why solvig this problem, Future situation

\begin{frame}{Motivation}{Introduction}

	\Columns{0.7}{0.3}{
		\Itemize{
			\item Donner un moyen aux utilisateurs de mesurer leur vulnérabilités
			\item Aider les utilisateurs à mieux configurer leur messagerie.
			\item Alerter les utilisateurs d'une nouvelle vulnérabilité.
			\item Sensibiliser les utilisateurs du niveau de diffusion des menaces.
		}
	}{
		\Figure{!htb}{1}{pi2.png}{Indice de confidentialité \cite{maximilien_privacyasaservice_2009}}
	}

\end{frame}

\subsection*{Défis à relever}
% Obstacles, How hard it is, what are the objectifs

\begin{frame}{Défis}{Introduction}

	\Columns{0.6}{0.4}{
		\Itemize{
			\item Recommander des mesures de sécurité personnalisés
				\Itemize{
					\item Nouveau mot de passe chaque période de temps
					\item Sécuriser l'échange avec des comptes vulnérables
					\item Adapter les permissions aux changements
				}
			\item Calculer la vulnérabilité de l'environnement social
				\Itemize{
					\item Calculer le niveau de vulnérabilité des interactions
					\item Calculer le niveau d'influence entre les utilisateurs.
					\item[] 
				}
			\item Calculer la vulnérabilité du chemin des messages
				\Itemize{
					\item Identification des serveurs MTA
					\item Attribuer une note de confiance à chaque serveur
					\item Calculer la confiance moyenne du chemin.
				}
		}
	}{
		\Figure{!htb}{1}{related.png}{Interaction sociale}
	}
\end{frame}

\begin{frame}[noframenumbering]{Défis}{Introduction}

	\Columns{0.6}{0.4}{
		\Itemize{
			\item Recommander des mesures de sécurité personnalisés
				\Itemize{
					\item Nouveau mot de passe chaque période de temps
					\item Sécuriser l'échange avec des comptes vulnérables
					\item Adapter les permissions aux changements
				}
			\item \blue{Calculer la vulnérabilité de l'environnement social}
				\Itemize{
					\item \blue{Calculer le niveau de vulnérabilité des interactions}
					\item \blue{Calculer le niveau d'influence entre les utilisateurs}
					\item[] 
				}
			\item Calculer la vulnérabilité du chemin des messages
				\Itemize{
					\item Identification des serveurs MTA
					\item Attribuer une note de confiance à chaque serveur
					\item Calculer la confiance moyenne du chemin.
				}
		}
	}{
		\FigureS{!htb}{1}{related.png}{Interaction sociale}
	}
\end{frame}

\subsection*{Contributions}
% How to overcome these obstacles
% How to verify and validate results ?

\begin{frame}{Contributions}{Introduction}

	\Columns{0.6}{0.4}{
		\Itemize{
			\item Estimation de l'indice de confidentialité social.
				\Itemize{
					\item Vulnérabilité individuelle -> Vulnérabilité sociale.
					\item Processus de diffusion de vulnérabilité.
					\item Relation entre confiance et vulnérabilité.
					\item Données: Émails de Enron \& Caliopen.
				}
		}
	}{
		\Figure{!htb}{0.6}{contagion_revised.jpg}{La vulnérabilité d'un utilisateur est la vulnérabilité de tous}
	}
\end{frame}










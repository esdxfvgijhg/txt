\section{Introduction} \label{sec:Introduction}

%Needs %%%%%%%%%%%%%%%%%%%%%%%%%%%%%%%%%%%%%%%%%%%%%%%%%%%%%%%%%%%%%%%%%%%% Context Current needs %%%%%%%%%%%%%%%%%%%%%%%%%%%%%%%%%%%%%%%%%%%%%%%%%%%%%%%%%%
With increasing frequency,
	communication between citizens and institutions occurs via some type of e-mechanisms,
	such as websites,
	email,
	and social media.
In particular,
	email platforms are widely being adopted because of their simplicity of use.
Due to the social aspect of these mechanisms,
	users are continuously affected by their friends' privacy vulnerability.
Users can take all the required measures to protect themselves from potential information leakage,
	but if their friends didn't respect the same measures,
	this indirectly harms their privacy concerns,
	especially when they grant a high-level of trust to them.

%Current bad state %%%%%%%%%%%%%%%%%%%%%%%%%%%%%%%%%%%%%%%%%%%%%%%% Problematic Current bad state of the research %%%%%%%%%%%%%%%%%%%%%%%%%%%%%%%%%%%%%%%%%%
Currently,
	available solutions address the privacy issues of users by measuring their vulnerability toward active attackers in low layer protocols (e.g.~HTTPS,
	SSL,
	PGP,
	IPsec,
	etc),
	or by suggesting new privacy policy settings of their applications.
These works are efficient to protect users from external vulnerabilities, but they appear very weak to protect users from (legitimate) information leakage between messaging services users.
Many other works, for example \cite{liu_framework_2010}, address this problem by measuring the users' privacy vulnerability individually without caring about the social context of the problem.
Few works \cite{zeng_trustaware_2014} \cite{b.s._privacy_2015} address this problem from a topological view of users' relationships during their social interactions.
%Future good state %%%%%%%%%%%%%%%%%%%%%%%%%%%%%%%%%%%%%%%%%%%%%%%% Motivation Future good state of the research %%%%%%%%%%%%%%%%%%%%%%%%%%%%%%%%%%%%%%%%%%%
Our work is motivated by the potential of privacy awareness frameworks to help users being conscious about the trustworthiness of their social interactions.

%Challenges %%%%%%%%%%%%%%%%%%%%%%%%%%%%%%%%%%%%%%%%%%%%%%%%%%% Challenges %%%%%%%%%%%%%%%%%%%%%%%%%%%%%%%%%%%%%%%%%%%%%%%%%%%%%%%%%%%%%%%%%%%%%%%%%%%%%%%%%%
Trust networks allow users to rate other users,
	they can put their level of trust in their interlocutors based on their own beliefs such as "Alice trust Bob as 0.8 in [0,1]" \cite{massa_trustaware_2007}.
Trust statements can then be aggregated in a single trust network representing the relationships between users \cite{massa_trustaware_2007}.
Trust metrics in our work are related to the relation strength between users such as the frequency of interactions,
	common interests,
	common friends, etc.
Trust metrics can also be related to the relationship closeness such as family,
	friends,
	colleagues or just unknown.
Based on such metrics and the topology of the interaction network,
	the system can suggest how many users are trustworthy based on different opinions of interlocutors,
	this suggestion represents their reputation.

Trust and reputation metrics are used in our work in order to study the relationship between them and users' privacy vulnerability.
Reputation concept refers to the extent to which a user is trustworthy.
This means that he plays a central role in preserving or revealing sensitive information of his interlocutors.
Reputation system collects,
	distributes and aggregates feedback about participants’ past behavior to allow users decide whom to trust and with whom to exchange sensitive information,
	users could then decide to not interact with those who are vulnerable to preserve their own privacy.

%Contribution %%%%%%%%%%%%%%%%%%%%%%%%%%%%%%%%%%%%%%%%%%%%%%%%%%%%%%%%%%%% Contribution %%%%%%%%%%%%%%%%%%%%%%%%%%%%%%%%%%%%%%%%%%%%%%%%%%%%%%%%%%%%%%%%%%%%
Messaging services users often exchange messages with a high number of users without caring about the vulnerability of their social environment.
In this paper,
	we deal with privacy issues by studying the impact of trust in preserving privacy.

%%%%%%%%%%%%%%%%%%%%%%%%%%%%%%%%%%%%%%%%%%%%%%%%%%%%%%%%% The structure %%%%%%%%%%%%%%%%%%%%%%%%%%%%%%%%%%%%%%%%%%%%%%%%%%%%%%%%%%%%%%%%%%%%%%%%%%%%%%%%%%%%
The remainder of this paper is organized as follows.
Section \ref{sec:Related work} elucidates summary of related works.
%Section II review current solutions,
In Section \ref{sec:Approach},
	we propose our vulnerability contagion process to reveal the social vulnerability of users.
Our experimentation with Enron dataset and our findings are presented in Section \ref{sec:Experimentation} and \ref{sec:Results exploitation} respectively.
%Section V shows the performance of our approach and discuss some applications
Finally,
	conclusion and future works are drawn in Section \ref{sec:Conclusions}.



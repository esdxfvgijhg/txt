\begin{poster}{
    grid=false,
    headerborder=open,           % Adds a border around the header of content boxes
    colspacing=1em,              % Column spacing
    bgColorOne=white,            % Background color for the gradient on the left side of the poster
    bgColorTwo=white,            % Background color for the gradient on the right side of the poster
    borderColor=white,           % Border color
    headerColorOne=violet,       % Background color for the header in the content boxes (left side)
    headerColorTwo=white,        % Background color for the header in the content boxes (right side)
    headerFontColor=white,       % Text color for the header text in the content boxes
    boxColorOne=white,           % Background color of the content boxes
    textborder=rounded,          %rectangle, % Format of the border around content boxes, can be: none, bars, coils, triangles, rectangle, rounded, roundedsmall, roundedright or faded
    eyecatcher=false,            % Set to false for ignoring the left logo in the title and move the title left
    headerheight=0.1\textheight, % Height of the header
    headershape=rounded,         % Specify the rounded corner in the content box headers, can be: rectangle, small-rounded, roundedright, roundedleft or rounded
    headershade=plain,
    headerfont=\Large\textsf,    % Large, bold and sans serif font in the headers of content boxes
    %textfont={\setlength{\parindent}{1.5em}}, % Uncomment for paragraph indentation
    linewidth=2pt                % Width of the border lines around content boxes
}{
    \includegraphics[scale=.3]{esiee}
}{
   \LARGE\textsf{Genetic Algorithm For LoRa Transmission Parameter Selection}
}{
    \sf\vspace{0.2em}\\
    Aghiles DJOUDI\Mark{1}\Mark{2}, Rafik ZITOUNI\Mark{2}, Nawel ZANGAR\Mark{1} and Laurent GEORGE\Mark{1}
    \vspace{0.3em}\\
    \small{
        \Mark{1} LIGM, UMR 8049, École des Ponts, UPEM, ESIEE Paris, CNRS,UPE, France\\
        \Mark{2} ECE Research Lab Paris, 37 Quai de Grenelle, 75015 Paris, France
        \vspace{0.3em}\\
    }
    Email:   aghiles.djoudi@esiee.fr, rafik.zitouni@ece.fr, nawel.zangar@esiee.fr, laurent.george@esiee.fr

}{
    \includegraphics[scale=.13]{esiee}
}


\headerbox{1. Introduction}{name=introduction,column=0,row=0, span=3}{
     The need of a new kind of wireless communication that could send data far away with limited resource constraints emerged recently to support IoT applications like smart building and smart environment monitoring.
 \textbf{LoRaWan} is one of this emerging wireless networks \cite{ayoub_internet_2019},
    it allows end-devices to reach the gateway in a range up to 5Km, 
    % it worth to mention that LoRaWan has a star topology so end-devices can communicate only 
Unlike other technologies LoRaWan is the best versatile solution to deploy IoT application in both urban and rural area where there is no communication infrastructure.

}

\headerbox{2. Parameter selection problem}{name=mcs,column=0,below=introduction,span=1}{
    The physical layer of LoRa technology (Semtech SX1276) has 4 parameters which make 6720 possible settings \cite{noura_interoperability_2018}:
    \begin{itemize}
        \item[\ding{224}] \textbf{SF:} Spreading factor [SF7 - SF12]
        \item[\ding{224}] \textbf{CR:} Coding rate [4/5 - 4/8]
        \item[\ding{224}] \textbf{BW:} Bandwidth [7.8Khz - 500Khz]
        \item[\ding{224}] \textbf{Tx:} Transmission power [-4dBm~+20dBm]
    \end{itemize}

    \begin{center}
        \includegraphics[width=\linewidth]{lorawan_parameters}
            iotnet.eu
    \end{center}
}

\headerbox{3. Genetic Algorithm}{name=model,column=0,below=mcs,span=1}{
    A genetic algorithm is a heuristic search that is used to deal with selection and ranking problems  \cite{vlahogianni_optimized_2005}.
    This algorithm reflects the process of natural selection where the fittest configurations are selected for reproduction in order to produce offspring of the next generation.
    \begin{itemize}
        \item[\ding{224}]  \textbf{Gene:} QoS metric.
        \item[\ding{224}]  \textbf{Chromosome:} QoS of one configuration.
        \item[\ding{224}]  \textbf{Population:} QoS of all configurations.
    \end{itemize}

        \begin{center}
        \includegraphics[width=\linewidth]{genetic_}
        towardsdatascience.com
    \end{center}
}



\headerbox{4. LoRaWan network}{name=image,span=2,column=1,below=introduction}{
    \begin{center}
        \includegraphics[width=.8\linewidth]{arch2}
    \end{center}
}


\headerbox{5. Algorithm}{name=screen,span=2,column=1,below=image}{
\medskip
\setlength{\columnsep}{-2.5cm}
    \begin{multicols}{2}
    \textbf{Definition:} stopping criteria, population size P, and mutation probability $p_{m}$\\
    \textbf{Generate} randomly the initial configurations \\
    \textbf{repeat:}\\
    . . . \textbf{for} each configuration do\\
    . . . . . . Train a model \& compute configuration's fitness\\
    . . . \textbf{end}\\
    . . . \textbf{for} each reproduction 1 ... P/2 do\\
    . . . . . . \textbf{Select:} 2 configurations based on fitness\\
    . . . . . . \textbf{Crossover:} Produce 2 child configurations\\
    . . . . . . \textbf{Mutate:} child configurations with $p_{m}$\\
    . . . \textbf{end}\\
    \textbf{until} stopping criterion are met\\
    \columnbreak
    \flushright
    \begin{tikzpicture}[node distance=1cm, every node/.style={fill=white, font=\sffamily}, align=center]
      \node (start)       [activityStarts]                     {Parameter initialization};
      \node (fitness)     [selection, below of=start]          {Fitness function};
      \node (crossover)   [selection, below of=fitness]        {Crossover};
      \node (mutation)    [selection, below of=crossover]      {Mutation};
      \node (selection)   [selection, below of=mutation]       {Survivor selection};
      \node (end)         [activityStarts, below of=selection] {Ranked selection list};
  
      \draw[->]     (start)     -- (fitness);
      \draw[->]     (fitness)   -- (crossover);
      \draw[->]     (crossover) -- (mutation);
      \draw[->]     (mutation)  -- (selection);
      \draw[->]     (selection) -- (end);
      \draw[->] (selection.east) to[bend right] (fitness.east);
    \end{tikzpicture}
\end{multicols}
}

\headerbox{6. Framework}{name=sea,span=2,column=1,below=screen}{
    \begin{center}
        \includegraphics[width=\linewidth]{genetic}
        The proposed scheme for LoRa transmission parameters selection based on GA, FL and Multi-Criteria Decision Making MCDM .
    \end{center}
    \medskip
    \Itemize{
        \item \textbf{Ongoing:} In order to generate all the required metrics of each LoRa configuration, we use ns3 simulator with 2 nodes and one gateway. The distance between nodes and the gateway is 1km.

    }



    % \begin{multicols}{2}
    %     \includegraphics[width=.8\linewidth]{generation}

    % \columnbreak
    %     \bigskip
    %     \
    %     \
    %     \
    %     \
    %     \
    %     \begin{center}
    %     \begin{tabular}{c|c|c|c}
    %         \textbf{Setup}   & \textbf{Selection error} & \textbf{Rank} & \textbf{Fitness} \\\hline
    %         \textbf{1}               & 0.9                      & 1             & 1.5               \\
    %         \textbf{2}               & 0.5                      & 3             & 4.5               \\
    %         \textbf{3}               & 0.7                      & 2             & 3                  \\
    %         \textbf{n}               & 0.5                      & 4             & 6                 \\
    %     \end{tabular}
    %     \end{center}

    % \end{multicols}

    % Results show that genetic algorithm select the configuration that match better the required QoS by the application.
    % % In fact,
    %     when we run an application that requires high quality of service,
    %     the algorithm select the configuration that gives large BW and hight data rate with minimum enrgy consumption.
    % When we run an application that requiers less QoS,
    %     the algorithm rank configuration whith sufficient BW and DR.

}

\headerbox{7. Discussion}{name=conclusion,column=1,below=sea,span=2,above=bottom}{

    \ding{224} \textbf{Advantages:}
        Genetic algorithms can manage data sets with many features.
        They don't need additional knowledge about the problem under study.
        In fact, such iterative algorithms require only the result of the last fitness value of the previous generation,
        % but they usually require much time to converge thus edge computing could solve this problem.
        % Such an approach could easily be deployed in network servers without additional configuration.

    \ding{224} \textbf{Conclusion:}
        LoRa transmission parameter selection problem by nature is a selection problem,
        thus, in our work we use genetic algorithm with a selection process to get the optimal subset of parameters that match better required QoS.
        Knowing the impact of each LoRa parameter on the output configuration still a big issue in research area.
        Our first results show that the transmission delay is more impacted by the SF and BW but less impacted by the CR.
    }

\headerbox{7. References}{name=references,column=0,span=1,below=model,above=bottom}{
    \small
    \renewcommand{\section}[2]{\vskip 0.05em} 
    % \bibliographystyle{unsrt}
    \printbibliography
}
\end{poster}


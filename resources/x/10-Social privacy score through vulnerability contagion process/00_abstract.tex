\begin{abstract}

% Statistics
The exponential usage of messaging services for communication raises many questions in privacy fields.
% Problem
Privacy issues in such services strongly depend on the graph-theoretical properties of users' interactions representing the real friendships between users.
One of the most important issues of privacy is that users may disclose information of other users beyond the scope of the interaction,
	without realizing that such information could be aggregated to reveal sensitive information.
% Challenges
Determining vulnerable interactions from non-vulnerable ones is difficult due to the lack of awareness mechanisms.
%Objective


% Contribution, methods
To address this problem, we analyze the topological relationships with the level of trust between users to notify each of them about their vulnerable social interactions.
Particularly,
	we analyze the impact of trusting vulnerable friends in infecting other users' privacy concerns by modeling a new vulnerability contagion process.
% Experimentation & results
Simulation results show that over-trusting vulnerable users speeds the vulnerability diffusion process through the network.
Furthermore,
	vulnerable users with high reputation level lead to a high convergence level of infection,
	this means that the vulnerability contagion process infects the biggest number of users when vulnerable users get a high level of trust from their interlocutors.
% The implication of this research in real life 15% 
This work contributes to the development of privacy awareness framework that can alert users of the potential private information leakages in their communications.

\end{abstract}


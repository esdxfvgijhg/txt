\section{Experimentation} \label{sec:Experimentation}

To evaluate our contagion process,
	we used Enron dataset in our experimentation.
There are many reasons for using Enron dataset to evaluate our vulnerability measure techniques.
First of all,
	it is probably the only actual corporate messaging service dataset available to the public.
Second,
	it contains all kind of emails "personal and official",
	so email logs can reveal the level of trust between users by studying the information flow in the network.
Finally,
	this dataset is similar to the kind of data collected for fraud detection or counter-terrorism,
	hence,
	it is a perfect test bed for testing the effectiveness of new vulnerability measure techniques.

The properties of the Enron dataset used for our experimentation are presented in table \ref{table:table2}.

\Table{|c|c|}{table2}{Enron dataset properties}{
	\ Parameter     & Value     \\\hline
	\ Nodes         & 958       \\
	\ Edges         & 6966      \\
	\ Diameter      & 958       \\
	\ Mean degree   & 2.413361  \\
	\ Edge density  & 0.00252   \\
	\ Modularity    & 0.654600  \\
	\ Mean distance & 3.042114  \\\hline
}

Due to visibility issues,
	we extract only important nodes given in \cite{shetty_discovering_2005}.
Our substantive interest in this experimentation is in how vulnerability moves through the network.
The inputs of our experimentation consist of a set of individual vulnerabilities of users in the network.
This set is generated randomly and is represented in the graph of Figure \ref{fig:local_.png},
	values of this set are between 0 and 1 and represent the extent to which users are exposed to different kind of attacks such as (phishing, spam, etc).
Unlike social vulnerability values,
	these values didn't take into account the vulnerability contagion process between users.
To evaluate how trust coefficient affects our outputs i.e., social privacy vulnerabilities,
	we variate the trust coefficient through 4 symmetric values 0.2, 0.4, 0.6 and 0.8.

%%%%%%%%%%%%%%%%%%%%%%%%%%%%%%%%%%%%%%%%%%%%%%%%%%%%%%%%%%%%%%%%%%%%%%%%%% Configuration %%%%%%%%%%%%%%%%%%%%%%%%%%%%%%%%%%%%%%%%%%%%%%%%%%%%%%%%%%%%%%%%%%%



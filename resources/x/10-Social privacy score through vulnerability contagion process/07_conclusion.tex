\section{Conclusions} \label{sec:Conclusions}

%%%%%%%%%%%%%%%%%%%%%%%%%%%%%%%%%%%%%%%%%%%%%%%%%%%%%%%%%%%%%%%%%%%%%%% Restate the main challenges %%%%%%%%%%%%%%%%%%%%%%%%%%%%%%%%%%%%%%%%%%%%%%%%%%%%%%%%
This paper studies the relationship between trust and reputation metrics in users' interactions and the social privacy vulnerability of users.
%%%%%%%%%%%%%%%%%%%%%%%%%%%%%%%%%%%%%%%%%%%%%%%%%%%%%%%%%%%%%%%%%%%%%%% Restate the main contribution %%%%%%%%%%%%%%%%%%%%%%%%%%%%%%%%%%%%%%%%%%%%%%%%%%%%%%
We observe that such metrics,
	severely affect users’ privacy in regard to their social relationships.
Trust coefficient in the social network graph plays an essential role in spreading vulnerability.
%%%%%%%%%%%%%%%%%%%%%%%%%%%%%%%%%%%%%%%%%%%%%%%%%%%%%%%%%%%%%%%%%%%%%%% Restate the main findings %%%%%%%%%%%%%%%%%%%%%%%%%%%%%%%%%%%%%%%%%%%%%%%%%%%%%%%%%%
Our experiment investigates the probability of infecting other's privacy by increasing the reputation of vulnerable users.
In this case,
	the number of nodes infected depends on the trust grant assigned to vulnerable users.
Our experiment reveals also that the social vulnerability of users could be extracted from their individual vulnerability by applying our vulnerability contagion process.
%%%%%%%%%%%%%%%%%%%%%%%%%%%%%%%%%%%%%%%%%%%%%%%%%%%%%%%%%%%%%%%%%%%% Future challenges current bad state %%%%%%%%%%%%%%%%%%%%%%%%%%%%%%%%%%%%%%%%%%%%%%%%%%
Privacy-preserving in online social network architectures should address this problem by discouraging trusting vulnerable friends.
%	trusting friends could be seen as:
	
%Moreover,
%	when providing communication obfuscation and identifiers integrity through random walk on the social network graph,
%	the OSN should guarantee the fast mixing property to the network.

%This can be done by ensuring the small world property of the social network graph,
%	and encouraging “long links” connecting different clusters together,
%	otherwise most of the random walk would be confined to the originating cluster.

%%%%%%%%%%%%%%%%%%%%%%%%%%%%%%%%%%%%%%%%%%%%%%%%%%%%%%%%%%%%%%%%%%%% Future contribution %%%%%%%%%%%%%%%%%%%%%%%%%%%%%%%%%%%%%%%%%%%%%%%%%%%%%%%%%%%%%%%%%%%
%As a future work,
%	we intend to study the impact of graph properties,
%	such as the diffusion degree,
%	betweenness,
%	and closeness on users trust grant function to know which social factors impact the trustworthiness of users.


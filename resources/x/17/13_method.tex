\subsection{Contagion process}
\begin{frame}{... (step 1)}{Methods}
	\Itemize{
		\item 
		\item 
	}
\end{frame}

%\note{
%	\begin{itemize}
%		\item Contenu:
%		\begin{itemize}
%			\item L’approche doit être soigneusement détaillée
%			\item Motiver les étapes, les hypothèses, le contexte
%			\item Dérouler un exemple si nécessaire
%			\item Illustrer par des schémas et figures
%			\item Se concentrer sur les aspects où l’approche apporte une contribution
%			\item Montrer comment on est différent de l’existant
%		\end{itemize}
%		\item Conseils:
%		\begin{itemize}
%			\item Présentation de l'approche utilisée pour résoudre le problème posé: approche(contrainte, parametre du probleme) = solution
%			\begin{itemize}
%				\item Justification du choix de l'approche
%				\item Description générale de l’approche comme une boite noir
%				\begin{itemize}
%					\item Entrées, Sorties, Contraintes, Hypothèses
%				\end{itemize}
%				\item Description détaillée de la solution du problème
%				\begin{itemize}
%					\item Description détaillée des étapes: paramètre -> étape1 -> étape2, ... -> solution
%					\item Modélisation des objets manipulés
%				\end{itemize}
%			\end{itemize}
%			\item Mise en œuvre des hypothèses
%			\item Description de la solution du problème
%		\end{itemize}
%		\item Conseils 2:
%		\begin{itemize}
%			\item Utilisez un exemple pour le dérouler tout au long des étapes de l’approche
%			\item Ne parlez dans ce chapitre que de votre travail, ce qu’ont fait les autres est dans l’état de l’art
%			\item Insistez sur les parties où vous apportez des contributions
%			\item Montrez comment votre travail est différent des autres
%			\item Montrez les modules/algorithmes pris de l’existant, ne réinventez pas la roue.
%			\item le lecteur doit pouvoir reproduire les résultats en appliquant la même approche.
%		\end{itemize}
%	\end{itemize}
%}

\begin{frame}{... (step 2)}{Methods}
	\Itemize{
		\item 
		\item 
	}
\end{frame}

\begin{frame}{... (step 3)}{Methods}
	\Itemize{
		\item 
		\item 
	}
\end{frame}

\begin{frame}{... (step 4)}{Methods}
	\Itemize{
		\item 
		\item 
	}
\end{frame}

\begin{frame}{Results}{Comparison}
	\begin{table}[h!]
	\scriptsize
		\begin{tabulary}{\textwidth}{L|L|L|L|L}
		\  &  &  &  &  \\\hline
		\  &  &  &  &  \\\hline
		\  &  &  &  &  \\\hline
		\  &  &  &  &  \\\hline
		\  &  &  &  &  \\\hline
		\end{tabulary}
	\caption{\label{tab:} }
	\end{table}
\end{frame}


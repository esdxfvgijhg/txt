\section{Background}

The goal of the implementation phase is to have a working prototype for future assessment.
To make the process of implementing the prototype possible,
	the first part of the implementation process will be to create a set of requirements.
When these are set,
	the process will continue by comparing the data from chapter 2 to find the candidates that fulfil the requirements.
After the technologies are selected,
	the process will continue with setting up the workspace,
	which includes the platform for development and the required tools to build,
	debug and test the prototype.
Finally,
	when these three steps have been performed,
	the next step will be to start with the actual prototype development.
	
\subsection*{Requirements}
As the time dedicated for development is limited,
	the requirements have to make sure that the development process does not run into any major obstacles.
All parts of the total prototype need to fit together seamlessly.
However,
	the hardware platform and the operating system are tied most closely together.
Therefore,
	they need requirements that complement each other,
	so that they can act as a platform for software development.
Naturally,
	both the hardware and the operating system requirements might have to be altered slightly to enable the best match.

\subsection{Hardware}
Each platform examined in chapter 2 has different strengths and weaknesses.
When looking at the MCU,
	OpenMote has a ARM Cortex-M3 which is more powerful compared to the other two alternatives:
	it features a 32bit 32MHz core with 32KB RAM and 512KB flash memory compared to the MSP430 16bit 25MHz core with ∼10KB/∼100KB memory configuration.
In terms of peripherals all three platforms are comparable,
	with similar amount of DAC,
	GPIO pins,
	and external busses.
All of the platforms have a temperature sensor and an accelerometer,
	but OpenMote also features an light/uv-light sensor and a voltage sensor built into the MCU’s ADC.
The MSB430 platform has a somewhat lower power consumption in active RF mode thanks to the less power-hungry sub-GHz transceiver.
On the other hand,
	the less powerful MSP430 MCU has a better deep sleep power consumption,
	but as the radio is not integrated in the chip as it is in the CC2538 SoC,
	that advantage is offset by the external transceiver.
Comparing the transceivers,
	there are two 2.4GHz models and one subGHz model;
	the sub-GHz CC1100 has a higher transmit power of 10dBm compared to 0dBm for CC2420 and 7dBm for CC2538.
Also the sensitivity is similar for all the alternatives but gives a slight advantage to CC2538 with -97dBm compared to -95dBm and -93dBm for CC2420 and CC1100.
Using these numbers and Friis range equation (equation 4.1) the range of each transceiver with a Fade margin (FM) of 20dB can be seen in figure 3.1.
The benefits of working with lower frequencies can clearly be seen as the theoretical range of CC1100 is almost 3 times longer than the 2.4GHz transceivers.

Adding all this information together,
	the choice of platform will land on OpenMote with the CC2538SoC.
It both has a MCU with more memory and better performance,
	and a transceiver with really good characteristics both in terms of energy consumption and range.
Also OpenMote is the only option that can act as a border router using OpenBase;
	it lets the SoC interface with USB,
	UART,
	JTAG,
	and Ethernet,
	which enables the standalone border router mode without the need to be connected to a computer or other hardware.
The OpenBattery extension lets the SoC operate as a node in a mesh network and provides a dual AAA battery slot connected to the PCB.

\subsubsection*{Resume}
• Transceiver with IEEE 802.15.4 or IEEE 802.15.1
• Integrated sensor/sensors
• MCU with low power mode under 5MicroA
• Wakeup from low power mode with timer
• Border router ability
• Joint Test Action Group (JTAG) support

\subsection{Operating system}

As a modern operating system can be compiled to match almost any hardware,
	the most important thing to have in mind is the out-of-the-box hardware support.
Only RIOT and Contiki have full support for the ARM Cortex-M3 of the considered operating systems and thus both TinyOS and freeRTOS are directly eliminated as developing the support would take too much time.
Compared to RIOT,
	Contiki also has full driver support for the sensors and transceiver,
	which should decrease the implementation time significantly.
When compiled into binary form,
	RIOT uses less RAM and ROM and thus probably is a bit faster compared to Contiki,
	which could be important if the application consumes much resources.
The lower memory usage might also give RIOT an advantage in being future-proof.
Contiki also has support for soft real-time scheduling compared to the hard real-time scheduling of RIOT;
	this is however not crucial,
	as the software that will be running on the OS does not have any hard real-time constraints.
Both RIOT and Contiki have support for 6LoWPAN but no support for either ZigBee or BLE;
	this is due to the fact that these are proprietary stacks.
Support could be added but would take some time to customize for the given OS.
What gives Contiki the largest advantage is that it also have border router software ready for deployment,
	which in the RIOT case would have to be developed.
All in all,
	as the project has such a limited time frame,
	Contiki will be selected as the OS;
	this,
	mainly because Contiki comes with most advantages time-wise;
	this choice means that the focus of the software development will be the creation of a test and evaluation system.

\subsubsection*{Resume}
• Support for the SoC/MCU and transceiver
• 6LoWPAN, ZigBee or BLE stack
• Soft real-time
• RAM and ROM footprint matching the hardware


\subsection{Communication protocol}


The OpenMote platform has a IEEE 802.15.4 transceiver and thus supports both ZigBee and 6LoWPAN;
	this means that BLE is not an option.
As ZigBee does not have full IPv6 support yet and is not integrated into Contiki,
	the natural choice is 6LoWPAN.
This choice will not only save some development time but also enables evaluation of the header compression.
As seen in figure 3.2,
	the 6LoWPAN stack in Contiki will replace the IP stack while maintaining the same functionality.
As the functionality is the same,
	TCP and HTTP will work with 6LoWPAN,
	but including them in the source increases the OS build size considerably.
On top of the UDP layer,
	Contiki also has a working implementation of CoAP that can be used for retrieving data from the nodes in a power efficient manner.
CoAP is a stateless protocol that uses the HTTP response headers to achieve a very low overhead in transmissions while using application level reliability methods to ensure packet delivery.

\subsubsection*{Resume}
• IPv6 addressing support
• Existing OS support
• Network type is mesh
• UDP


\subsection{Workspace and tools}

The ARM Cortex-M3 chip that OpenMote and CC2538 is built upon requires the GCC ARM Embedded compiler.
This tool-chain is free and runs on both Linux,
	OSX,
	and Windows;
	however,
	there is no bundled development application so a secondary application for programming is needed.
In Windows,
	there are several Integrated Development Environments (IDEs) such as IAR Workbench ARM [30],
	Code Composer Studio [31] and the Eclipse plug-in ARM DS-5 [32, 33];
	these IDEs use various proprietary toolchains and have a price tag ranging from free to several thousand SEK.
Most of the IDEs also have a code size restriction for the free versions.
To minimize the costs,
	the development machine development machine used in this project will run Ubuntu 14.04 LTS,
	the used tool-chain is GCC ARM Embedded,
	and Geany is used as the code development application.
To analyse the network traffic in real-time,
	the open source tool Wireshark is used together with a IEEE 802.15.4 packet sniffer.
Together with a laptop,
	the packet sniffer will grant the ability to traverse the mesh network and analyse the network in real-time as it is seen by the nodes.


